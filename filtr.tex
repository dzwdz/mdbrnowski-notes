\documentclass{scrartcl}
\usepackage[pretty,polish]{mystd}
\title{Funkcje tworzące i filtracja pierwiastkami jedności}
\author{Michał Dobranowski}
\date{czerwiec 2022 \\ v1.2}

\begin{document}
    \maketitle
    \subsection*{Wymagania wstępne}
    Czytelnik powinien znać matematykę na poziomie liceum oraz mieć podstawową wiedzę na temat liczb zespolonych, w tym znać wzór Eulera. Ponadto powinien wiedzieć, czym jest wzór dwumienny Newtona oraz znać podstawy teorii grup (chociaż bez tego ostatniego też można się obejść).
    \tableofcontents
    \eject

\section{Pierwiastki jedności w ciele liczb zespolonych}
    \begin{definition}
        \label{d:root}
        Pierwiastek $n$-tego stopnia z jedynki to liczba $\omega \in \CC$ spełniąjaca równanie $\omega^n = 1$.
    \end{definition}

    \begin{lemma}
        \label{l:roots_sum}
        Niech $\omega \neq 1$ będzie pierwiastkiem $n$-tego stopnia z jedynki. Wtedy
        $$ \omega + \omega^2 + \ldots + \omega^n = 0. $$
    \end{lemma}
    \begin{proof}
        Z równości $\omega^n = 1$ mamy
        $$ (1 - \omega)(\omega + \omega^2 + \ldots + \omega^n) = \omega - \omega^{n+1} = 0. $$
        Z założenia $(1 - \omega) \neq 0$, więc otrzymujemy tezę.
    \end{proof}

    \begin{definition}
        \label{d:primitive_root}
        Pierwiastek pierwotny $n$-tego stopnia z jedynki to taka liczba $\omega_n^k = e^{2\pi i k / n}$, że $n$~i~$k$ są względnie pierwsze.
    \end{definition}

    \begin{lemma}
        \label{l:roots_group}
        Zbiór wszystkich pierwiastków jedności $n$-tego stopnia tworzy grupę cykliczną ze względu na mnożenie. Generatorem tej grupy jest każdy pierwiastek pierwotny $n$-tego stopnia z jedności.
    \end{lemma}
    \begin{proof}
        Na podstawie definicji \ref{d:root} i \ref{d:primitive_root} oraz własności potęgowania łatwo zauważyć, że taka grupa jest izomorficzna z grupą $(\NN_n, +)$, czyli grupą addytywną modulo~$n$, której generatorem jest każda liczba względnie pierwsza z $n$.
    \end{proof}

    \begin{remark*}[dla Czytelnika nieobeznanego z teorią grup]
        Dla danej liczby $z \in \CC$ zdefiniujmy $P(z) = \{z, z^2, z^3, \ldots\}$. Lemat \ref{l:roots_group} mówi o tym, że dla każdego pierwiastka pierwotnego $n$-tego stopnia z jedności $\omega_n^k$ zbiór $P(\omega_n^k)$ jest zbiorem wszystkich pierwiastków $n$-tego stopnia z jedności.

        Przykładowo, $P(\omega_6^5) = \{\omega_6^5, \omega_6^4, \omega_6^3, \omega_6^2, \omega_6^1, 1\}$, ale $P(\omega_6^2) = \{\omega_6^2, \omega_6^4, 1\}$.
    \end{remark*}

    \begin{lemma}
        \label{l:roots_plus_one_prod}
        Jeśli $\omega$ jest pierwiastkiem pierwotnym $n$-tego stopnia z jedności, to
        $$ (1 + a\omega)(1 + a\omega^2)\cdots(1 + a\omega^n) = 1 - (-a)^n $$
    \end{lemma}
    \begin{proof}
        Zbiór pierwiastków jedności $n$-tego stopnia jest również zbiorem miejsc zerowych wielomianu $x^n - 1$, więc
        $$ x^n - 1 = (x - \omega)(x - \omega^2)\cdots(x - \omega^n). $$
        Podstawmy $x = -\frac{1}{a}$ i pomnóżmy obustronnie przez $a^n$.
        $$ (-1)^n - a^n = (-1 - a\omega)(-1 - a\omega^2)\cdots(-1 - a\omega^n) $$
        $$ (-1)^n - a^n = (-1)^n(1 + a\omega)(1 + a\omega^2)\cdots(1 + a\omega^n) $$
        $$ (1 + a\omega)(1 + a\omega^2)\cdots(1 + a\omega^n) = \frac{(-1)^n - a^n}{(-1)^n} = 1 - (-a)^n $$
    \end{proof}

    Szczególnie często będziemy korzystać z faktu, że dla nieparzystego $n$ zachodzi
    $$ (1 + \omega)(1 + \omega^2)\cdots(1 + \omega^n) = 2. $$

    \subsection{Zadania}
    \begin{problem}[Berkeley Math Tournament 2015, Analysis 8]
        Niech $\omega$ będzie pierwotnym pierwiastkiem jedności $7$ stopnia. Oblicz
        $$ \prod_{k = 0}^6(1 + \omega^k + \omega^{2k}). $$
        \begin{hint}
            Rozłóż trójmian na iloczyn dwumianów.
        \end{hint}
        \begin{answer}
            Najpierw rozłóżmy i oznaczmy szukany iloczyn
            $$ P = \prod_{k = 0}^6(1 + \omega^k + \omega^{2k}) = -\prod_{k = 0}^6(\phi - \omega)(\tau - \omega), $$
            gdzie $\phi, \tau$ to rozwiązania równania $x^2 - x - 1 = 0$.

            Z lematu \ref{l:roots_group} oraz rozumownia podobnego do dowodu lematu \ref{l:roots_plus_one_prod} wiemy, że
            $$ P = -(\phi^7 - 1)(\tau^7 - 1). $$

            Teraz zauważmy, że dla $\phi$ i $\tau$ spełnione jest $x^2 = x + 1$. Za pomocą tej równości upraszczamy $x^7 - 1$ do momentu, aż uzyskamy wyrażenie liniowe (obliczenia pominiemy).
            $$ x^7 - 1 = 13x + 7 $$
            Używając tego wyniku oraz wzorów Viète'a: $\phi + \tau = 1, \phi\tau = -1$, mamy
            $$ P = -(13\phi + 7)(13\tau + 7) = -(-169 + 91 + 49) = 29. $$
        \end{answer}
    \end{problem}

    \begin{problem}[Berkeley Math Circle 2000, MC5/2]
        Pewien prostokąt jest pokryty kostkami domina: pionowymi $1 \times n$ oraz poziomymi $1 \times m$, gdzie $n, m \in \NN$. Wykaż, że ten prostokąt można też pokryć wyłącznie jednym z wymienionych typów kostek domina.
        \begin{hint}
            W każde pole prostokąta wpisz pewną liczbę zespoloną.
        \end{hint}
        \begin{answer}
            Niech wymiary prostokąta to $a \times b$. Musimy teraz pokazać, że $m \mid a$ lub $n \mid b$. Niech $\omega = e^{2\pi i/m}$ oraz $\xi = e^{2\pi i/n}$. Dzielimy prostokąt na $ab$ pól; w pole o współrzędnych $(x, y)$ wpisujemy liczbę $\omega^x\xi^y$.

            Możemy obliczyć sumę liczb w prostokącie:
            $$ (\omega + \omega^2 + \ldots + \omega^a)(\xi + \xi^2 + \ldots + \xi^b) = \omega\xi\frac{\omega^a - 1}{\omega - 1}\frac{\xi^b - 1}{\xi - 1} $$

            Zauważmy (na podstawie lematu \ref{l:roots_sum}), że suma liczb w każdej kostce domina jest równa~$0$. Tak więc również suma liczb w całym prostokącie jest równa $0$. Stąd wynika, że $\omega^a = 1$ lub $\xi^b = 1$, czyli $m \mid a$ lub $n \mid b$.
        \end{answer}
    \end{problem}

    \begin{problem}[Berkeley Math Tournament 2015, Analysis 7]
        Oblicz
        $$ \sum_{k = 0}^{37} (-1)^k\binom{75}{2k}. $$
        \begin{hint}
            Użyj wzoru Newtona.
        \end{hint}
        \begin{answer}
            Na podstawie wzoru Newtona, $\sum_{k = 0}^{75}\binom{75}{x}x^k = (1 + x)^{75}$. Wystarczy więc obliczyć
            \begin{align*}
                \Re\left((1 + i)^{75}\right) &= \sqrt{2}^{75} \cdot \cos\left(75\cdot\frac{\pi}{4}\right) \\
                                              &= \sqrt{2}^{75} \cdot \cos\left(\frac{3\pi}{4}\right) \\
                                              &= -2^{37}.
            \end{align*}
        \end{answer}
    \end{problem}

\section{Funkcje tworzące}
    \begin{definition}
        Funkcja tworząca $G(x)$ dla ciągu $(g_n)_{n\geq 0}$ to funkcja taka, że
        $$ G(x) = \sum_{n = 0}^\infty g_nx^n. $$
    \end{definition}

    Oczywiście ciąg nieskończony może równie dobrze być od pewnego miejsca zerowy -- wtedy śmiało można uznać go za skończony i odpowiednio zmodyfikować definicję funkcji tworzącej. Rozpatrzmy następujący przykład:
    \begin{example}
        Rzucamy dwoma standardowymi, sześciennymi kośćmi. Takie zdarzenie losowe jest określane przez pewien \vocab{rozkład prawdopodobieństwa}\footnote{czyli funkcję, która każdemu wynikowi przypisuje jego prawdopodobieństwo.}, dotyczący sumy wyrzuconych oczek, przykładowo $P(2) = \frac{1}{36}, P(7) = \frac{1}{6}$. Rozważ, czy da się stworzyć dwie \textbf{niestandardowe} kości (lecz dalej sześcienne, z całkowitą dodatną liczbą oczek) tak, aby rozkład prawdopodobieństwa się nie zmienił.
    \end{example}
    \begin{solution}
        Łatwo zauważyć, że taką standardową kostkę możemy opisać jako
        $$ \sD(x) = x + x^2 + x^3 + x^4 + x^5 + x^6. $$
        Wtedy funkcja tworzącą ciąg $(g_n)$, gdzie $P(n) = \frac{g_n}{36}$ jest dana po prostu jako $\sD(x)^2$. Rozłóżmy ten wielomian na czynniki.
        \begin{align*}
            \sD(x)^2 &= \left(x + x^2 + x^3 + x^4 + x^5 + x^6\right)^2 \\
                    &= x^2\left(1 + x + x^2 + x^3 + x^4 + x^5\right)^2 \\
                    &= x^2\left(1 + x^3\right)^2\left(1 + x + x^2\right)^2 \\
                    &= x^2\left(1 + x\right)^2\left(1 - x + x^2\right)^2\left(1 + x + x^2\right)^2
        \end{align*}
        Szukane kości możemy oznaczyć jako $\sA(x)$ i $\sB(x)$. Teraz po kolei będziemy przedstawiać warunki stawiane w zadaniu.
        \begin{enumerate}
            \item Rozkład prawdopodobieństwa ma być taki sam, więc $\sA(x)\sB(x) = \sD(x)^2$. Możemy więc zapisać
            $$ \sA = x^{a_1}\left(1 + x\right)^{a_2}\left(1 - x + x^2\right)^{a_3}\left(1 + x + x^2\right)^{a_4} $$
            $$ \sB = x^{b_1}\left(1 + x\right)^{b_2}\left(1 - x + x^2\right)^{b_3}\left(1 + x + x^2\right)^{b_4} $$
            gdzie dla każdego $j\in{1,2,3,4}$ zachodzi $a_j + b_j = 2$.
            \item Liczby oczek na ściankach mają być całkowite dodatnie, więc obie te funkcje są podzielne przez $x$. To implikuje $a_1 = b_1 = 1$.
            \item Kości mają być sześcienne, więc $\sA(1) = \sB(1) = 6$. Z tego wynika, że $2^{a_2}3^{a_4} = 2^{b_2}3^{b_4} = 6$, więc $a_2 = a_4 = b_2 = b_4 = 1$.
        \end{enumerate}
        Skoro chcemy, by kości były różne od $\sD$, to jest już tylko jedna opcja: $a_3 = 2, b_3 = 0$ (z~dokładnością do kolejności). Wtedy otrzymujemy dwie kości o wartościach $(1, 3, 4, 5, 6, 8)$ i $(1, 2, 2, 3, 3, 4)$.
    \end{solution}

    Zauważmy, że jeśli chcielibyśmy rozważać więcej niż dwie kości, żaden z warunków by się nie zmienił, jedynie końcowy wniosek byłby inny: dla $n$ kości mamy $\binom{2n - 1}{n} - 1$ możliwych rozwiązań (zamiast dwóch w wersji z dwiema kośćmi).

    Zobaczyliśmy właśnie jeden z przykładów użycia funkcji tworzących. W następnej sekcji powiemy o jednej szczególnej technice ich wykorzystania w kombinatoryce.

\section{Filtracja pierwiastkami jedności}
    \begin{theorem}[Filtracja pierwiastkami jedności, \textit{Root of unity filter}]
        \label{t:filter}
        Niech $\omega = e^{2\pi i/n}$, gdzie $n \in \NN$. Dla wielomianu $F(x) = a_0 + a_1x + a_2x^2 + \ldots$ zachodzi
        $$ a_0 + a_n + a_{2n} + \ldots = \frac{1}{n}\left(F(1) + F(\omega) + \ldots + F(\omega^{n - 1})\right) $$
    \end{theorem}
    \begin{proof}
        Niech
        $$ s_k = 1 + \omega^k + \ldots + \omega^{(n - 1)k}. $$
        Jeśli $n \mid k$, to $\omega^k = 1$, więc $s_k = 1 + 1 + \ldots + 1 = n$. W przeciwnym przypadku $s_k = \frac{1 - \omega^{nk}}{1 - \omega^k} = 0$.
        Z tego powodu mamy
        $$ F(1) + F(\omega) + \ldots + F(\omega^{n - 1}) = a_0s_0 + a_1s_1 + a_2s_2 + \ldots $$
        $$ F(1) + F(\omega) + \ldots + F(\omega^{n - 1}) = n(a_0 + a_n + a_{2n} + \ldots), $$
        co, po obustronnym podzieleniu przez $n$, kończy dowód.
    \end{proof}

    \begin{example}
        Oblicz $$ A = \sum_{k = 0}^{33} \binom{100}{3k}. $$
    \end{example}
    \begin{solution}
        Ze wzoru Newtona wynika, że funkcją tworzącą ciągu $\left(\binom{100}{0}, \binom{100}{1}, \cdots, \binom{100}{n}\right)$ jest
        $$ f(x) = (x + 1)^{100}. $$
        Musimy obliczyć co trzeci wyraz tego ciągu, więc, korzystając z filtrowania pierwiastkami jedności (twierdzenie \ref{t:filter}), weźmy $\omega = e^{2\pi i /3}$ i zapiszmy $A$ jako
        $$ A = \frac{1}{3}\left(f(1) + f(\omega) + f(\omega^2)\right). $$

        Oczywiście $f(1) = 2^{100}$. Możemy sprowadzić $\omega + 1$ oraz $\omega^2 + 1$ do postaci wykładniczej i wtedy policzyć sumę, ale ładniejszym wyjściem będzie wykorzystanie lematu \ref{l:roots_sum} i~zapisanie
        $$ (\omega + 1)^{100} = (-\omega^2)^{100} = \omega^{200} = \omega^2 $$
        oraz
        $$ (\omega^2 + 1)^{100} = (-\omega)^{100} = \omega^{100} = \omega. $$
        Ostatecznie
        $$ A = \frac{1}{3}\left(2^{100} + \omega^2 + \omega\right) = \frac{2^{100} - 1}{3}. $$

    \end{solution}

    \begin{example}
        Znajdź liczbę takich podzbiorów zbioru $S = \{1\ldots 2000\}$, że suma elementów każdego z~nich jest podzielna przez $5$.
    \end{example}
    \begin{solution}
        Najpierw zauważmy, że funkcja
        $$ f(x) = (1 + x)(1 + x^2)\cdots(1 + x^{2000}) $$
        jest funkcją tworzącą ciągu $(a_n)$, gdzie $a_n$ to liczba podzbiorów zbioru $S$, których suma jest równa $n$.

        Teraz oznaczmy $\omega = e^{2\pi i / 5}$. Korzystając z filtrowania pierwiastkami jedności możemy stwierdzić, że szukaną liczbą jest
        $$ \frac{1}{5}\sum_{j=0}^4 f(\omega^j). $$
        Łatwo zobaczyć, że $f(1) = 2^{2000}$. Na podstawie lematów \ref{l:roots_group} i \ref{l:roots_plus_one_prod} wnioskujemy, że
        $$ f(\omega^j) = \left((1 + \omega)(1 + \omega^2)(1 + \omega^3)(1 + \omega^4)(1 + \omega^5)\right)^{400} = 2^{400} $$
        dla $j \in \{1, 2, 3, 4\}$ -- czyli dla pierwiastków pierwotnych.

        Podsumowując, szukaną liczbą jest
        $$ \frac{1}{5}\left(2^{2000} + 4\cdot 2^{400}\right) = \frac{2^{2000} + 2^{402}}{5}. $$
    \end{solution}

    \subsection{Dwuargumentowa funkcja tworząca}
    Nic nie stoi na przeszkodzie, żebyśmy stworzyli dwuargumentową funkcję tworzącą, jeśli akurat jest to nam potrzebne. Wtedy ciąg będzie niejako dwuwymiarowy. Dostosowanie definicji do tego przypadku nie powinno Czytelnikowi sprawić trudności.

    \begin{example}[IMO 1995/6]
        Niech $p$ będzie nieparzystą liczbą pierwszą. Znajdź liczbę takich $p$-elementowych podzbiorów $A$ zbioru $\{1,2,\ldots,2p\}$, że suma elementów z $A$ jest podzielna przez $p$.
    \end{example}
    \begin{solution}
        Użyjmy następującej funkcji tworzącej ciąg $(a_{n,m})$:
        $$ f(x, y) = (1 + xy)(1 + x^2y)\cdots(1 + x^{2p}y). $$
        Łatwo zauważyć, że będziemy chceli znaleźć sumę współczynników przy wyrazach postaci $x^s y^p$ gdzie $p\mid s$. Oznaczając $\omega = e^{2\pi i / p}$ i korzystając z filtrowania pierwiastkami jedności mamy
        $$ \sum_{s,k\geq 0} a_{s,k}y^k = \frac{1}{p}\left(f(\omega, y) + f(\omega^2, y) + \cdots + f(\omega^p, y)\right). $$
        Na mocy lematów \ref{l:roots_group} i \ref{l:roots_plus_one_prod} otrzymujemy
        \begin{align*}
            f(\omega^j, y) &= (1 + \omega^j y)(1 + \omega^{2j}y)\cdots(1 + \omega^{2pj}y) \\
                           &= \left((1 + \omega^j y)(1 + \omega^{2j}y)\cdots(1 + \omega^{pj}y)\right)^2 \\
                           &= \left((1 + \omega y)(1 + \omega^2 y)\cdots(1 + \omega^p y)\right)^2 \\
                           &= (1 + y^p)^2
        \end{align*}
        dla $j \in\{1, 2, \ldots, p-1\}$. Zauważając, że $f(1, y) = (1 + y)^{2p}$, mamy teraz
        $$ \sum_{s,k\geq 0} a_{s,k}y^k = \frac{1}{p}\left((1 + y)^{2p} + (p - 1)(1 + y^p)^2\right). $$
        Teraz wystarczy policzyć sumę współczynników przy $y^p$, która wynosi
        $$ \frac{1}{p}\left(\binom{2p}{p} + 2(p - 1)\right). $$
    \end{solution}

    Wykorzystane w ten sposób funkcje tworzące mają właściwie nieograniczone możliwości. Modyfikując powyższe zdanie, możemy na przykład podać liczbę zbiorów, których suma elementów jest podzielna przez $p > 3$, a liczba elementów podzielna przez $3$. Wystarczy drugi raz zastosować filtrację. Oznaczając wynik przez $\sR$ oraz biorąc $\xi = e^{2\pi i / 3}$ mamy
    $$ \sR = \frac{1}{3}\cdot\frac{1}{p}\left(\sum_{j = 1}^3\left((1 + \xi^j)^{2p} + (p - 1)(1 + \xi^{pj})^2\right)\right). $$
    Wielokrotnie korzystając z lematów \ref{l:roots_sum} i \ref{l:roots_group} upraszczamy
    \begin{align*}
        \sR &= \frac{1}{3p}\left(\sum_{j = 1}^3 (1 + \xi^j)^{2p} + (p - 1)\sum_{j = 1}^3 (1 + \xi^{pj})^2\right) \\
          &= \frac{1}{3p}\left(2^{2p} + (-\xi^2)^{2p} + (-\xi)^{2p} + (p - 1)\left(2^2 + (-\xi^2)^2 + (-\xi)^2\right)\right) \\
          &= \frac{1}{3p}\left(2^{2p} + \xi^p + \xi^{2p} + (p - 1)(4 + \xi + \xi^2)\right) \\
          &= \frac{1}{3p}\left(2^{2p} - 1 + 3(p - 1)\right) \\
          &= \frac{1}{3p}\left(2^{2p} + 3p - 4\right).
    \end{align*}

    \subsection{Zadania}
    \begin{problem}
        Znajdź liczbę podzbiorów zbioru $S = \{1,2,\ldots,2070\}$, których suma elementów jest podzielna przez $9$.
        \begin{hint}
            Uwaga, $9$ nie jest liczbą pierwszą.
        \end{hint}
        \begin{answer}
            Niech $a$ będzie takim ciągiem, że $a_k$ jest liczbą podbiorów zbioru $S$ o sumie równej~$k$. Funkcją tworzącą tego ciągu jest
            $$ f(x) = (1 + x)(1 + x^2)\cdots(1 + x^{2070}). $$

            Niech $\omega = e^{2\pi i/9}$. Używając filtrowania pierwiastkami jedności możemy stwierdzić, że odpowiedzią na pytanie jest liczba
            $$ \frac{1}{9}\sum_{j = 1}^9 f(\omega^j). $$

            Zauważmy, że
            $$ f(\omega^j) = \left((1 + \omega^j)\cdots(1 + \omega^{9j})\right)^{230}, $$
            więc, na podstawie lematu \ref{l:roots_plus_one_prod}, mamy $f(\omega^j) = 2^{230}$ dla każdego $j$ względnie pierwszego~z~$9$, czyli $j\in\{1,2,4,5,7,8\}$.

            Ponadto, $\omega^3$ oraz $\omega^6$ są pierwiastkami pierwotnymi trzeciego stopnia z jedynki, więc dla $j\in\{3,6\}$ zachodzi
            $$ f(\omega^j) = \left((1 + \omega^j)(1 + \omega^{2j})(1 + \omega^{3j})\right)^{690} = 2^{690}. $$

            Na koniec obliczamy $f(1) = 2^{2070}$, więc
            $$ \frac{1}{9}\sum_{j = 1}^9 f(\omega^j) = \frac{1}{9}\left(2^{2070} + 2\cdot 2^{690} + 6\cdot 2^{230}\right). $$
        \end{answer}
    \end{problem}

    \begin{problem}[IMC 1999, B2]
        Wykonujemy $n$ rzutów standardową kostką. Oblicz, jakie jest prawdopodobieństwo, że suma wyrzuconych oczek będzie podzielna przez $5$.
        \begin{answer}
            Oznaczmy zbiór interesujących nas zdarzeń jako $A$. Korzystając z filtrowania pierwiastkami jedności, weźmy $\omega = e^{2\pi i / 5}$ i obliczmy
            $$ |A| = \frac{1}{5}\sum_{j = 1}^5 f(\omega^j)^n, $$
            gdzie $f(x) = x + x^2 + x^3 + x^4 + x^5 + x^6$ jest funkcją tworzącą standardowej kostki. Na mocy lematów \ref{l:roots_sum} i \ref{l:roots_group} mamy
            $$ |A| = \frac{1}{5}\left(6^n + \omega^n + \omega^{2n} + \omega^{3n} + \omega^{4n} + \omega^{5n}\right). $$
            Jeśli $5\mid n$, to
            $$ |A| = \frac{1}{5}\left(6^n + 4\right), $$
            w przeciwnym przypadku
            $$ |A| = \frac{1}{5}\left(6^n - 1\right). $$
            Moc zbioru zdarzeń elementarnych wynosi $6^n$, więc mamy
            $$ P(A) = \begin{cases}\frac{1}{5} + \frac{4}{5\cdot 6^n}, & \text{ jeśli } 5\mid n \\
                                   \frac{1}{5} - \frac{1}{5\cdot 6^n}, & \text{ w przeciwnym wypadku.} \end{cases} $$
        \end{answer}
    \end{problem}

    \begin{problem}[Concurs ,,Traian Lalescu'' 2003]
        Znajdź liczbę takich naturalnych liczb $n$-cyfrowych, że każda z nich jest podzielna przez $3$ oraz każda jej cyfra zawiera się w~zbiorze $\{2, 3, 7, 9\}$.
        \begin{answer}
            Najpierw zauważmy, że funkcją tworzącą ciągu $(a_k)_{k\geq 0}$, gdzie $a_k$ jest liczbą $n$-cyfrowych liczb złożonych z cyfr $\{2, 3, 7, 9\}$ o sumie cyfr równej $k$ jest
            $$ f(x) = \left(x^2 + x^3 + x^7 + x^9\right)^n. $$
            Szukamy takich liczb, których suma cyfr jest podzielna przez $3$, więc możemy wziąc $\omega = e^{2\pi i / 3}$ i, korzystając z filtrowania pierwiastkami jedności, obliczyć
            $$ \frac{1}{3}\left(4^n + f(\omega) + f(\omega^2)\right) $$
            $$ = \frac{1}{3}\left(4^n + \left(\omega^2 + 1 + \omega + 1\right)^n + \left(\omega + 1 + \omega^2 + 1\right)^n\right) $$
            Z lematu \ref{l:roots_sum} otrzymujemy równość z
            $$ \frac{1}{3}\left(4^n + 2\right). $$
        \end{answer}
    \end{problem}

    Warto jeszcze spojrzeć na (to znaczy zrobić) zadania wymienione w \cite{ref:NICE}.

\begin{thebibliography}{9}
    \bibitem{ref:Kessler} \textbf{Complex Numbers}, Andre Kessler.
        \url{https://web.archive.org/http://web.mit.edu/~akessler/www/lectures/complex.pdf}
    \bibitem{ref:Aggrawal} \textbf{Root of unity filter}, Shweta Aggrawal.
        \url{https://math.stackexchange.com/a/3213153/842118}
        (odpowiedź na Math Stack Exchange)
    \bibitem{ref:Abel} \textbf{Multivariate Generating Functions and Other Tidbits}, Zachary R. Abel.
        \url{http://zacharyabel.com/papers/Multi-GF_A06_MathRefl.pdf}
    \bibitem{ref:Wilf} \textbf{generatingfunctionology}, Herbert Wilf.
        \url{https://www2.math.upenn.edu/~wilf/gfology2.pdf}
    \bibitem{ref:NICE} \textbf{NICE Journal 2}, NICE Committee
        \url{https://www.nicecontest.xyz/static/media/NJ2.942d32c0.pdf#page=12}
\end{thebibliography}

\newpage
\section{Wskazówki}
    \makehints

\section{Rozwiązania}
    \makeanswers

\end{document}