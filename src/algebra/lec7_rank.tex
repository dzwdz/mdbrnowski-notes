\begin{definition}
    \label{d:rank}
    Rząd macierzy to maksymalna liczba liniowo niezależnych wektorów kolumnowych tej macierzy. Rząd macierzy $A$ oznaczamy przez $\rank (A)$.
\end{definition}

\begin{theorem}
    Dla każdej macierzy $A \in \sM_{m\times n}$ zachodzi
    \[ \rank(A) = \rank(A^T). \]
\end{theorem}
\begin{proof}
    Niech $r = \rank A$. Weźmy macierz $L \in \sM_{m\times r}$, która zawiera $r$ liniowo niezależnych wektorów kolumnowych macierzy $A$. Każda kolumna macierzy $A$ jest oczywiście kombinacją liniową kolumn macierzy $L$, czyli $A = LZ$ dla pewnej macierzy współczynników $Z \in \sM_{r\times n}$.

    Z faktu \ref{f:AB^T = B^T A^T} mamy $A^T = Z^TL^T$, więc $\rank (A^T)$ jest ograniczony z góry przez $\rank(Z^T)$, który jest ograniczony z góry przez $r$ (liczbę kolumn macierzy $Z^T$), więc
    \[ \rank(A^T) \leq \rank(A). \]
    Powtarzamy argument dla macierzy $A^T$ i, wykorzystując fakt $(A^T)^T = A$, otrzymujemy
    \[ \rank(A) \leq \rank(A^T), \]
    więc
    \[ \rank(A) = \rank(A^T). \]
\end{proof}

Możemy teraz zmodyfikować definicję rzędu (\ref{d:rank}): jest to maksymalna liczba liniowo niezależnych wektorów kolumnowych lub wektorów wierszowych. Oczywiście zachodzi nierówność $\rank (A_{m\times n}) \leq n, m$.

\begin{definition}
    Macierz schodkowa to macierz, której pierwsze niezerowe elementy (schodki) kolejnych niezerowych wierszy znajdują się w coraz dalszych kolumnach, a~wiersze zerowe umieszczone są najniżej.
\end{definition}

\begin{fact}
    Rząd macierzy schodkowej jest równy liczbie jej schodków.
\end{fact}

\begin{definition}
    Operacje elementarne na macierzach to:
    \begin{itemize}
        \item zamiana miejscami wierszy (kolumn) macierzy,
        \item dodanie do wiersza (kolumny) kombinacji liniowej pozostałych wierszy (kolumn),
        \item pomnożenie wiersza przez niezerowy skalar.
    \end{itemize}
    Jeśli macierz $B$ można otrzymać z macierzy $A$ za pomocą operacji elementarnych, to będziemy oznaczać $A \sim B$.
\end{definition}

\begin{fact}
    Rząd macierzy nie zmienia się pod wpływem operacji elementarnych.
\end{fact}
\begin{proof}
    Wynika z twierdzenia \ref{t:linear independence}.
\end{proof}

Każdą macierz można łatwo doprowadzić do postaci schodkowej, za pomocą metody \vocab{eliminacji Gaussa}, która polega na stosowaniu operacji elementarnych na wierszach, ,,pozbywając się'' niezerowych elementów z dolnego trójkąta. W ten sposób można odczytać jej rząd oraz, z pomocą wniosku \ref{c:determinant in triangular matrix}, obliczyć jej wyznacznik (jeśli jest kwadratowa)\footnote{Warto jednak zwrócić uwagę, że przy obliczaniu wyznacznika lepiej nie mnożyć wierszy i nie zamieniać ich miejscami, bo te operacje wpływają na wyznacznik.}.

\begin{example}
    Obliczyć wyznacznik macierzy
    \[ \begin{bmatrix}
        1 & 3 & 1 \\
        1 & 1 & -1 \\
        3 & 11 & 6
    \end{bmatrix}. \]
\end{example}
\begin{solution}
    \[ \begin{vmatrix}
        1 & 3 & 1 \\
        1 & 1 & -1 \\
        3 & 11 & 6
    \end{vmatrix} \overset{\substack{r_2 - r_1 \\ r_3 - 3r_1}}{=} \begin{vNiceMatrix}[create-medium-nodes]
        \CodeBefore[create-cell-nodes]
            \begin{tikzpicture}[name suffix = -medium]
                \node [highlight = (1-1) (1-3)] {} ;
                \node [highlight = (2-2) (2-3)] {} ;
                \node [highlight = (3-2) (3-3)] {} ;
            \end{tikzpicture}
        \Body
        1 & 3 & 1 \\
        0 & -2 & -2 \\
        0 & 2 & 3
    \end{vNiceMatrix} \overset{r_3 + r_2}{=} \begin{vNiceMatrix}[create-medium-nodes]
        \CodeBefore[create-cell-nodes]
            \begin{tikzpicture}[name suffix = -medium]
                \node [highlight = (1-1) (1-3)] {} ;
                \node [highlight = (2-2) (2-3)] {} ;
                \node [highlight = (3-3) (3-3)] {} ;
            \end{tikzpicture}
        \Body
        1 & 3 & 1 \\
        0 & -2 & -2 \\
        0 & 0 & 1
    \end{vNiceMatrix} = -2 \]
\end{solution}

\begin{theorem}
    \label{t:rank = max order of nonzero minor}
    Rząd macierzy $A$ jest równy największemu ze stopni niezerowych minorów tej macierzy.
\end{theorem}
\begin{proof}
    Jeśli pewna podmacierz kwadratowa $k \times k$ jest nieosobliwa (czyli minor jest niezerowy), to jej kolumny są liniowo niezależne, czyli $\rank A \geq k$. Natomiast jeśli każda podmacierz wymiaru $(k + 1) \times (k + 1)$ jest osobliwa (czyli minory są zerowe), to żaden podzbiór $k + 1$ wektorów kolumnowych nie jest liniowo niezależny, więc $\rank A < k + 1$. Z tego wynika, że $\rank A = k$.
\end{proof}

\begin{example}
    Obliczyć rząd macierzy
    \[ A = \begin{bmatrix}
        1 & 3 & 4 \\
        3 & 4 & 1 \\
        1 & 2 & 7 \\
        3 & 5 & -1
    \end{bmatrix}. \]
\end{example}
\begin{solution}
    Obliczmy minor
    \[ \begin{vmatrix}
        1 & 3 & 4 \\
        3 & 4 & 1 \\
        3 & 5 & -1
    \end{vmatrix} = \begin{vmatrix}
        1 & 3 & 4 \\
        3 & 4 & 1 \\
        0 & 1 & -2
    \end{vmatrix} = (-8) + 0 + 12 - 0 - 1 - (-18) = 21 \neq 0. \]
    Ten minor jest niezerowy i jednocześnie ma największy stopień (bo nie wykreśliliśmy żadnej kolumny), więc na mocy twierdzenia \ref{t:rank = max order of nonzero minor} $\rank A = 3$.

    Dla pewności można pokazać również inną metodę --- eliminację Gaussa:
    \[ \begin{bmatrix}
        1 & 3 & 4 \\
        3 & 4 & 1 \\
        1 & 2 & 7 \\
        3 & 5 & -1
    \end{bmatrix} \sim \begin{bmatrix}
        1 & 3 & 4 \\
        0 & -5 & -11 \\
        0 & -1 & 3 \\
        0 & -4 & -13
    \end{bmatrix} \sim \begin{bmatrix}
        1 & 3 & 4 \\
        0 & 0 & -26 \\
        0 & -1 & 3 \\
        0 & 0 & -25
    \end{bmatrix} \sim \begin{bmatrix}
        1 & 3 & 4 \\
        0 & -1 & 3 \\
        0 & 0 & 1 \\
        0 & 0 & 0 \\
    \end{bmatrix} \]
    \[ \therefore \rank A = \rank \begin{bNiceMatrix}[create-medium-nodes]
        \CodeBefore[create-cell-nodes]
            \begin{tikzpicture}[name suffix = -medium]
                \node [highlight = (1-1) (1-3)] {} ;
                \node [highlight = (2-2) (2-3)] {} ;
                \node [highlight = (3-3) (3-3)] {} ;
            \end{tikzpicture}
        \Body
        1 & 3 & 4 \\
        0 & -1 & 3 \\
        0 & 0 & 1 \\
        0 & 0 & 0 \\
    \end{bNiceMatrix} = 3. \]
\end{solution}