\begin{definition}
    \label{d:vector_space}
    Przestrzeń wektorowa (liniowa) nad ciałem $(K, \oplus, \otimes)$ to struktura $(V, K, +, \cdot)$, gdzie
    \begin{enumerate}[noitemsep,nolistsep]
        \item $(V, +)$ jest grupą abelową,
        \item działanie $\cdot : K \times V \lthen V$ jest zewnętrzne
        \item działanie $\cdot$ jest rozdzielne względem działania $+$, to znaczy
            \[ \dforall{u, v \in V} \dforall{\alpha \in K} \alpha\cdot(u + v) = (\alpha \cdot u) + (\alpha \cdot v), \]
        \item zachodzi ,,rozdzielność'' działania $\cdot$ względem $+$ i $\oplus$, to znaczy
            \[ \dforall{v \in V} \dforall{\alpha, \beta \in K} (\alpha \oplus \beta) \cdot v = (\alpha \cdot v) + (\beta \cdot v), \]
        \item zachodzi ,,łączność'' działań $\cdot$ i $\otimes$, to znaczy
            \[ \dforall{v \in V} \dforall{\alpha, \beta \in K} (\alpha \otimes \beta) \cdot v = \alpha \cdot (\beta \cdot v), \]
        \item jedynka z ciała $(K, \oplus, \otimes)$ jest elementem neutralnym również dla działania $\cdot$, to znaczy
            \[ \dforall{v \in V} \mathbf{1} \cdot v = v. \]
    \end{enumerate}
\end{definition}

Elementy zbioru $V$ nazywamy \vocab{wektorami}, a zbioru $K$ -- \vocab{skalarami}. Często zamiast przestrzeni $(V, K, +, \cdot)$ piszemy o przestrzeni $V$, a zamiast symboli $\oplus, \otimes$ piszemy po prostu $+, \cdot$. Element neutralny dodawania wektorów to wektor zerowy $\ol{0}$.

\begin{example}
    Przestrzenią wektorową nad ciałem liczb rzeczywistych jest struktura $(\RR^n, \RR, +, \cdot)$, często oznaczana jako $\RR^n(\RR)$, gdzie
    \begin{itemize}[noitemsep,nolistsep]
        \item $(x_1, x_2, \ldots, x_n) + (y_1, y_2, \ldots, y_n) = (x_1 + y_1, \ldots, x_n + y_n)$,
        \item $\alpha \cdot (x_1, x_2, \ldots, x_n) = (\alpha x_1, \alpha x_2, \ldots, \alpha x_n)$.
    \end{itemize}
\end{example}

\begin{example}
    Jeśli przez $\RR[x]_n$ oznaczymy zbiór wielomianów rzeczywistych o stopniu równym co najwyżej $n$, to struktura \[(\RR[x]_n, \RR, +, \cdot)\]
    będzie przestrzenią liniową.
\end{example}

\begin{theorem}
    W przestrzeni liniowej $(V, K, +, \cdot)$ dla każdych $u, v \in V$ oraz $\alpha, \beta \in K$ zachodzą następujące własności:
    \begin{enumerate}[noitemsep,nolistsep]
        \item $\mathbf{0} \cdot v = \ol{0}$,
        \item $\alpha \cdot \ol{0} = \ol{0}$,
        \item $(-\alpha) \cdot v = - (\alpha \cdot v)$,
        \item $\alpha \cdot (-v) = - (\alpha \cdot v)$,
        \item $\alpha \cdot v = \ol{0} \iff (\alpha = \mathbf{0} \lor v = \ol{0})$,
        \item $\alpha \cdot u = \alpha \cdot v \implies u = v$, dla $\alpha \neq \mathbf{0}$,
        \item $\alpha \cdot v = \beta \cdot v \implies \alpha = \beta$, dla $v \neq \ol{0}$.
    \end{enumerate}
\end{theorem}
\begin{proof}
    W dowodach wszystkich własności posługujemy się wyłącznie definicją przestrzeni wektorowej (\ref{d:vector_space}), wektora zerowego oraz poprzednimi w kolejności udowadnianymi własnościami.
    \begin{enumerate}[noitemsep,nolistsep]
        \item $\begin{aligned}[t]
            v + \mathbf{0} \cdot v &= \mathbf{1} \cdot v + \mathbf{0} \cdot v = (\mathbf{1} + \mathbf{0}) \cdot v = v = v + \ol{0} \\
            \therefore \mathbf{0} \cdot v &= \ol{0}
            \end{aligned}$
        \item $\begin{aligned}[t]
            \alpha \cdot \ol{0} &= \alpha \cdot (\ol{0} + \ol{0}) = \alpha \cdot \ol{0} + \alpha \cdot \ol{0} \\
            \therefore \ol{0} &= \alpha \cdot \ol{0}
            \end{aligned}$
        \item $\begin{aligned}[t]
            \ol{0} = \alpha \cdot v - (\alpha \cdot v) &\text{ oraz } \ol{0} = \mathbf{0} \cdot v = (\alpha - \alpha) \cdot v = \alpha \cdot v + (-\alpha) \cdot v \\
            \therefore -(\alpha \cdot v) &= (-\alpha) \cdot v
            \end{aligned}$
        \item $\begin{aligned}[t]
            \ol{0} = \alpha \cdot v - (\alpha \cdot v) &\text{ oraz } \ol{0} = \alpha \cdot \ol{0} = \alpha \cdot (v - v) = \alpha \cdot v + \alpha \cdot (-v) \\
            \therefore -(\alpha \cdot v) &= \alpha \cdot (-v)
            \end{aligned}$
        \item implikacja $\Leftarrow$ (konieczność) trywialna; implikacja $\Rightarrow$ (dostateczność) wynika z tego, że jeśli założymy, że $\alpha \neq \mathbf{0}, v \neq \ol{0}$, to mamy
            \[ \alpha \cdot (u + v) = \alpha \cdot u + \alpha \cdot v = \alpha \cdot u. \]
            Mnożąć przez $a^{-1}$ (które istnieje, bo $(K, +, \cdot)$ jest ciałem) otrzymujemy
            \[ u + v = u, \]
            a dodając obustronnie $-u$ (które istnieje z definicji \ref{d:vector_space}) dochodzimy do sprzeczności z założeniem
            \[ v = \ol{0}. \]
        \item dowód analogiczny do dowodu lematu \ref{l:cancellation_property},
        \item dowód analogiczny do dowodu lematu \ref{l:cancellation_property}.
    \end{enumerate}
\end{proof}

\begin{definition}
    Podprzestrzeń liniowa $(U, K, +, \cdot)$ to taka struktura, że
    \begin{enumerate}[noitemsep,nolistsep]
        \item $(V, K, +, \cdot)$ jest przestrzenią liniową oraz $U \subset V, U \neq \emptyset$,
        \item $\dforall{u, v \in U} (u + v) \in U$,
        \item $\dforall{\alpha \in K} \dforall{u \in U} (\alpha \cdot u) \in U$.
    \end{enumerate}
\end{definition}

\begin{fact}[Równoważna charakterystyka podprzestrzeni]
    Dwa ostatnie warunki z powyższej definicji są równoważne warunkowi:
    \[ \dforall{\alpha, \beta \in K} \dforall{u, v \in V} \alpha \cdot u + \beta \cdot v \in U. \]
\end{fact}
\begin{proof}
    Implikacja w jedną stroną jest trywialna, w drugą stronę można ją udowodnić przez stwierdzenie, że każdy wektor ma wektor przeciwny (bo z definicji \ref{d:vector_space} $(V, +)$ jest grupą abelową) oraz że pod $\alpha, \beta$ można podstawić $\mathbf{1}$ (i znowu użyć definicji \ref{d:vector_space}).
\end{proof}

\begin{definition}
    Kombinacja liniowa wektorów $v_1, v_2, \ldots, v_n$ to wektor
    \[ \alpha_1v_1 + \alpha_2v_2 + \ldots + \alpha_nv_n, \]
    gdzie skalary $\alpha_1, \alpha_2, \ldots, \alpha_n$ nazywamy współczynnikami tej kombinacji.
\end{definition}

\begin{definition}
    Wektory $v_1, v_2, \ldots, v_n$ są liniowo niezależne, jeśli dla każdego ciągu współczynników $\alpha$ zachodzi implikacja
    \[ \alpha_1v_1 + \alpha_2v_2 + \ldots + \alpha_nv_n = \ol{0} \implies \alpha_1, \alpha_2, \ldots, \alpha_n = 0. \]
\end{definition}
Mówimy również, że wektory są liniowo zależne, jeśli nie są liniowo niezależne.

\begin{example}
    W przestrzeni wektorowej $\RR^3(\RR)$ weźmy wektory
    \[ u = (3, 2, -1), v = (1, -2, 1), w = (1, 1, 1). \]
    Rozwiązujemy układ równań $\alpha u + \beta v + \gamma w = \ol{0} \implies$
    \begin{equation*}
        \begin{cases}
            3\alpha + \beta + \gamma = 0 \\
            2\alpha - 2\beta + \gamma = 0 \\
            -\alpha + \beta + \gamma = 0
        \end{cases} \implies
        \begin{cases}
            4\alpha = 0 \\
            2\alpha - 2\beta + \gamma = 0 \\
            -\alpha + \beta + \gamma = 0
        \end{cases} \implies
        \begin{cases}
            \alpha = 0 \\
            - 2\beta + \gamma = 0 \\
            \beta + \gamma = 0
        \end{cases} \implies
        \begin{cases}
            \alpha = 0 \\
            \beta = 0 \\
            \gamma = 0
        \end{cases}
    \end{equation*}
    pokazując, że wektory $u, v, w$ są liniowo niezależne.
\end{example}

\begin{theorem}
    \label{t:linear independence}
    Wektory $v_1, \ldots, v_n$ są liniowo zależne wtedy i tylko wtedy, gdy przynajmniej jeden jest kombinacją liniową pozostałych.
\end{theorem}
\begin{proof}
    Jeśli istnieje taki ciąg $\alpha_1, \alpha_2, \ldots, \alpha_n$, że $\{\alpha_1, \ldots, \alpha_n\} \neq \{0\}$ oraz
    \[ \alpha_1v_1 + \alpha_2v_2 + \ldots + \alpha_nv_n = \ol{0}, \]
    to bez starty ogólności możemy przyjąć, że $\alpha_n \neq 0$. Równoważnie przekształcamy równość do postaci
    \[ \alpha_1v_1 + \alpha_2v_2 + \ldots + \alpha_{n-1}v_{n-1} = -\alpha_nv_n \]
    \[ \frac{-\alpha_1}{\alpha_n}v_1 + \frac{-\alpha_2}{\alpha_n}v_2 + \ldots + \frac{-\alpha_{n-1}}{\alpha_n}v_{n-1} = v_n, \]
    więc otrzymujemy równoważność między założeniem i stwierdzeniem, że $v_n$ jest kombinacją liniową wektorów $\alpha_1, \ldots, \alpha_{n-1}$.
\end{proof}

\begin{theorem}
    \label{t:explicit coefficients}
    Jeśli wektory $v_1, v_2, \ldots, v_n$ są liniowo niezależne oraz wektor $u$ jest kombinacją liniową tych wektorów, to współczynniki tej kombinacji są wyznaczone jednoznacznie.
\end{theorem}
\begin{proof}
    Weźmy takie ciągi $(\alpha_n)$ i $(\beta_n)$, że
    \begin{align*}
        u = \alpha_1v_1 + \alpha_2v_2 + \ldots + \alpha_nv_n \\
        u = \beta_1v_1 + \beta_2v_2 + \ldots + \beta_nv_n
    \end{align*}
    Mamy
    \[ u - u = \ol{0} = (\alpha_1 - \beta_1)v_1 + (\alpha_2 - \beta_2)v_2 + \ldots + (\alpha_n - \beta_n)v_n \]
    co, skoro $v_1, v_2, \ldots, v_n$ są liniowo niezależne, dowodzi, że dla każdego $i$ zachodzi $\alpha_i - \beta_i = 0$, więc ciągi $(\alpha_n)$ i $(\beta_n)$ są równe.
\end{proof}

\begin{definition}
    Powłoka liniowa zbioru $A \subset V, A \neq \emptyset$, gdzie $V$ jest przestrzenią wektorową nad ciałem $K$ to zbiór
    \[ \Lin A = \{v = \alpha_1v_1 + \alpha_2v_2 + \ldots + \alpha_kv_k : \alpha_i \in K, v_i \in A\} \]
\end{definition}

$\Lin A$ jest podprzestrzenią przestrzeni $A$ nazywaną podprzestrzenią generowaną przez zbiór $A$. Dla danego zbioru $A$ mówimy, że \vocab{rozpina} on przestrzeń wektorową $V$, jeśli $\Lin A = V$.

\begin{definition}
    Baza $B$ przestrzeni wektorowej $V$ to taki zbiór, że $\Lin B = V$ oraz wszystkie wektory w $B$ są liniowo niezależne.
\end{definition}

$B$ jest bazą danej przestrzeni liniowej wtedy i tylko wtedy, gdy $B$ jest maksymalnym (w sensie inkluzji) zbiorem wektorów liniowo niezależnych oraz wtedy i tylko wtedy, gdy $B$ jest minimalnym (w sensie inkluzji) zbiorem wektorów rozpinających. Przestrzeń $\{\ol{0}\}$ nie ma bazy.

\begin{theorem}
    \label{t:equinumerous bases}
    Każde dwie bazy danej przestrzeni wektorowej są równoliczne.
\end{theorem}
\begin{proof}
    Weźmy dwie bazy $A, B$ przestrzeni liniowej $V$ oraz niech $|A| = k$. Załóżmy przeciwnie, że $|B| > k, B = \{b_1, b_2, \ldots, b_k, \ldots\}$. Skoro $A$ jest bazą przestrzeni $V$, to każdy wektor ze zbioru $B$ jest kombinacją liniową wektorów ze zbioru $A$, czyli
    \[ b_1 = \alpha_1a_1 + \alpha_2a_2 + \ldots + \alpha_ka_k. \]
    Bez straty ogólności możemy założyć, że $\alpha_1 \neq 0$ (ponieważ wszystkie nie mogą być zerowe). Wtedy
    \[ a_1 = \frac{1}{\alpha_1}b_1 + \frac{-\alpha_2}{\alpha_1}a_2 + \ldots + \frac{-\alpha_k}{\alpha_1}a_k, \]
    a więc $a_1$ jest kombinacją liniową wektorów ze zbioru $A \cup \{b_1\} \setminus \{a_1\}$. Wektory z tego zbioru oczywiście rozpinają całą przestrzeń liniową $V$ oraz są liniowo niezależne (wszystkie $a_2, a_3, \ldots, a_k$ są liniowo niezależne, a $b_1$ jest liniowo niezależny od nich, ponieważ założyliśmy, że $\alpha_1 \neq 0$). Z tego powodu zbiór $A \cup \{b_1\} \setminus \{a_1\}$ jest bazą. Kontynuujemy rozumowanie, pokazując, że zbiór
    \[ A \cup \{b_1, b_2, \ldots, b_k\} \setminus \{a_1, a_2, \ldots a_k\} = \{b_1, b_2, \ldots, b_k\} \]
    jest bazą. Z tego powodu każdy wektor $b_{k+1}, b_{k+2}, \ldots$ jest liniowo zależny od $\{b_1, \ldots, b_k\}$, więc dochodzimy do sprzeczności z założeniem, że $B$ jest bazą.
\end{proof}

\begin{definition}
    Wymiar $\dim V$ przestrzeni wektorowej $V$ to liczność bazy tej przestrzeni. Jeśli $V = \{\ol{0}\}$, to $\dim V = 0$.
\end{definition}

\begin{example}
    Przestrzeń $(\RR^n, \RR, +, \cdot)$ jest przestrzenią skończenie wymiarową z
    \[ \dim\RR^n = n, \]
    natomiast $(\sF(\RR,\RR), \RR, +, \cdot)$ jest przestrzenią nieskończenie wymiarową, więc
    \[ \dim (\sF(\RR,\RR)) = \infty. \]
\end{example}

\begin{definition}
    Reper bazowy to baza, w której ustaliliśmy kolejność wektorów.
\end{definition}

Jeśli $B = (e_1, e_2, \ldots, e_n)$ jest reperem bazowym przestrzeni wektorowej $V$, to dla dowolnego wektora
\[ v = \alpha_1e_1 + \alpha_2e_2 + \ldots + \alpha_ne_n \]
skalary $\alpha_i$ nazwiemy \vocab{współrzędnymi} wektora $v$ w bazie $B$ i zapiszemy
\[ v = [\alpha_1, \alpha_2, \ldots, \alpha_n]_B. \]

\begin{definition}
    Baza kanoniczna to reper bazowy przestrzeni $\RR^n(\RR)$, w którym
    \[ B_k = \left((1,0,0,\ldots,0), (0,1,0,\ldots,0), (0,0,0,\ldots,1)\right). \]
\end{definition}

Łatwo uzasadnić, że jeśli $\dim V = n$, to każdy zbiór $n + 1$ wektorów jest liniowo zależny, a każdy zbiór $n$ wektorów jest liniowo niezależny wtedy i tylko wtedy, gdy generuje przestrzeń $V$.

\begin{theorem}
    Niech $V$ będzie przestrzenią skończenie wymiarową, a $U$ jej podprzestrzenią. Wówczas
    \[ \dim U = \dim V \quad\iff\quad U = V. \]
\end{theorem}
Implikacja w lewą stronę jest trywialna, pokażemy implikację w prawo.
\begin{proof}
    Jeśli weźmiemy pewną bazę $B$ przestrzeni $U$ i zachodzi warunek $\dim U = \dim V$, to, zgodnie z tym co powiedzieliśmy wcześniej, jest ona również bazą przestrzeni $V$, ponieważ $U \subset V$, z czego wynika teza.
\end{proof}

\begin{example}
    Przykłady przestrzeni wektorowych wraz z wymiarami:
    \begin{itemize}
        \item dla $(\KK^n, \KK, +, \cdot)$ przy $\KK = \RR, \CC, \ldots$ mamy $\dim \KK^n = n$,
        \item dla $(\CC^n, \RR, +, \cdot)$ mamy $\dim \CC^n = 2n$.
    \end{itemize}
\end{example}

\begin{definition}
    \label{d:sum of spaces}
    Suma podprzestrzeni $V_1, V_2$ przestrzeni $V$ to zbiór
    \[ V_1 + V_2 = \{v = v_1 + v_2 : v_1 \in V_1, v_2 \in V_2\}. \]
\end{definition}

\begin{fact}
    Jeśli $V_1, V_2$ są podprzestrzeniami przestrzeni $V$, to $V_1 \cap V_2$ jest podprzestrzenią przestrzeni $V$.
\end{fact}

\begin{remark}
    O ile $V_1 \cap V_2$ oraz $V_1 + V_2$ (z definicji) są przestrzeniami, o tyle już $V_1 \cup V_2$ na ogół nią nie jest, więc nie będziemy raczej używać tego zapisu.
\end{remark}

\begin{definition}
    Suma prosta $V_1 \oplus V_2$ dwóch podprzestrzeni przestrzeni $V$ to taka suma $V_1 + V_2$, że zachodzi warunek
    \[ \dforall{v \in V_1 + V_2} \dexistsone{v_1 \in V_1} \dexistsone{v_2 \in V_2} v = v_1 + v_2. \]
\end{definition}

\begin{theorem}
    Suma dwóch podprzestrzeni jest sumą prostą wtedy i tylko wtedy, gdy ich częścią wspólną jest zbiór $\{\ol{0}\}$.
\end{theorem}
\begin{proof}
    Jeśli część wspólna dwóch podprzestrzeni jest równa $\{\ol{0}\}$, to ich bazy są rozłączne, a więc teza wynika z twierdzenia \ref{t:explicit coefficients}.
\end{proof}

\begin{definition}
    Przestrzeń uzupełniająca $V_2$ podprzestrzeni $V_1$ przestrzeni $V$ to taka przestrzeń, że
    \[ V_1 \oplus V_2 = V. \]
\end{definition}

\begin{fact}
    Dla każdej podprzestrzeni dowolnej przestrzeni istnieje przestrzeń uzupełniająca.
\end{fact}

\begin{theorem}
    Dla skończenie wymiarowych podprzestrzeni $V_1, V_2$ przestrzeni wektorowej $V$ zachodzi
    \[ \dim(V_1 + V_2) = \dim V_1 + \dim V_2 - \dim(V_1 \cap V_2), \]
    a w szczególności
    \[ V = V_1 \oplus V_2 \implies \dim V = \dim V_1 + \dim V_2. \]
\end{theorem}
\begin{proof}
    Możemy wziąć bazę $B$ przestrzeni $V$ oraz bazy $B_1, B_2$ odpowiednio podprzestrzeni $V_1, V_2$ takie, że $B_1, B_2 \subset B$. Oczywistym jest, że
    \[ |B_1 \cup B_2| = |B_1| + |B_2| - |B_1 \cap B_2|, \]
    więc z definicji sumy podprzestrzeni (\ref{d:sum of spaces}) i twierdzenia \ref{t:equinumerous bases} wynika teza.
\end{proof}