Układ $m$ równań liniowych z $n$ niewiadomymi $x_1, \ldots, x_n$ w postaci
\begin{equation}
    \label{eq:linear system}\begin{cases}
    a_{11}x_1 + a_{12}x_2 + \ldots + a_{1n}x_n &= b_1 \\
    a_{21}x_1 + a_{22}x_2 + \ldots + a_{2n}x_n &= b_2 \\
    \cdots \\
    a_{m1}x_1 + a_{m2}x_2 + \ldots + a_{mn}x_n &= b_m \\
\end{cases}\end{equation}
możemy reprezentować jako równanie macierzy. Macierz
\[ A = \begin{bNiceMatrix}
    a_{11} & a_{12} & \Cdots & a_{1n} \\
    a_{21} & a_{22} & \Cdots & a_{2n} \\
    \Vdots & \Vdots & \Ddots & \Vdots \\
    a_{m1} & a_{m2} & \Cdots & a_{mn}
\end{bNiceMatrix} \]
nazywamy \vocab{macierzą główną} (macierzą współczynników) układu \ref{eq:linear system}, a macierze
\[ B = \begin{bNiceMatrix}
    b_1 \\ b_2 \\ \Vdots \\ b_m
\end{bNiceMatrix}, \qquad X = \begin{bNiceMatrix}
    x_1 \\ x_2 \\ \Vdots \\ x_n
\end{bNiceMatrix} \]
nazywamy odpowiednio \vocab{kolumną wyrazów wolnych} oraz \vocab{kolumną niewiadomych}. Połączenie macierzy $A$ i $B$
\[ \begin{bNiceArray}{cSc}A & B\end{bNiceArray}  = \begin{bNiceArray}{cccIc}
    a_{11} & \Cdots & a_{1n} & b_1 \\
    \Vdots & \Ddots & \Vdots & \Vdots \\
    a_{m1} & \Cdots & a_{mn} & b_m
\end{bNiceArray} \]
jest \vocab{macierzą uzupełnioną} tego układu. Wtedy układ \ref{eq:linear system} zapisujemy macierzowo jako
\[ A \cdot X = B. \]

\begin{definition}
    Układ jednorodny to taki układ równań liniowych, że kolumna wyrazów wolnych jest zerowa.
\end{definition}

\begin{definition}
    Układ jest:
    \begin{itemize}
        \item \vocab{oznaczony}, jeśli ma jedno rozwiązanie,
        \item \vocab{nieoznaczony}, jeśli ma więcej niż jedno rozwiązanie,
        \item \vocab{sprzeczny}, jeśli nie ma rozwiązań.
    \end{itemize}
\end{definition}

\begin{definition}
    Układ kwadratowy to układ równań liniowych, w którym liczba niewiadomych jest równa liczbie równań (czyli macierz główna jest kwadratowa).
\end{definition}

\begin{definition}
    Układ Cramera to układ kwadratowy, w którym wyznacznik macierzy głównej jest niezerowy, $\det A \neq 0$.
\end{definition}

\begin{theorem}[Cramera]
    \label{t:Cramer}
    Jeśli dany układ jest układem Cramera, to jest oznaczony oraz
    \[ x_j = \frac{D_{x_j}}{\det A}, \]
    gdzie $D_{x_j}$ jest wyznacznikiem macierzy powstałej przez zastąpienie $j$-tej kolumny macierzy głównej $A$ kolumną wyrazów wolnych $B$.
\end{theorem}
\begin{proof}
    Pierwsze stwierdzenie jest prawdziwe, ponieważ jeśli $\det A \neq 0$, to kolumny tworzą bazę pewnej przestrzeni liniowej, do której należy kolumna $B$, więc na mocy twierdzenia \ref{t:explicit coefficients} jest ona jednoznacznie wyznaczona przez kombinację liniową kolumn z~$A$~(a współczynniki tej kombinacji liniowej są właśnie kolumną niewiadomych $X$).

    Oznaczając $j$-tą kolumną $A$ jako $\mathbf{a_j}$ oraz $B = \mathbf{b}$, na mocy poprzedniego akapitu równanie
    \[ x_1\mathbf{a_1} + x_2\mathbf{a_2} + \ldots + x_n\mathbf{a_n} = \mathbf{b} \]
    spełnia dokładnie jeden wektor $\mathbf{x}$. Zatem
    \[ D_{x_j} = \det(\mathbf{a_1}, \ldots, \mathbf{b}, \ldots, \mathbf{a_n}) = \det(\mathbf{a_1}, \ldots, \sum_{i=1}^n x_i\mathbf{a_i}, \ldots, \mathbf{a_n}). \]
    Z własności wyznaczników (\ref{t:determinant properties}) wynika, że wyznacznika nie zmieni odjęcie od pewnej kolumny innej kolumny przemnożonej przez skalar (nawet zerowy), więc
    \[ D_{x_j} = \det(\mathbf{a_1}, \ldots, x_j\mathbf{a_j}, \ldots, \mathbf{a_n}) = x_j\det(\mathbf{a_1}, \ldots, \mathbf{a_j}, \ldots, \mathbf{a_n}) = x_j \det A, \]
    \[ \therefore x_j = \frac{D_{x_j}}{\det A}. \]
\end{proof}

\begin{theorem}[Kroneckera-Capellego]
    \label{t:Kronecker-Cappelli}
    Układ $AX = B$ ma co najmniej jedno rozwiązanie wtedy i tylko wtedy, gdy
    \[ \rank(A) = \rank(\begin{bNiceArray}{cSc}A & B\end{bNiceArray}). \]
\end{theorem}
\begin{proof}
    Oznaczając $j$-tą kolumną $A$ jako $\mathbf{a_j}$ oraz $B = \mathbf{b}$, mamy
    \[ x_1\mathbf{a_1} + x_2\mathbf{a_2} + \ldots + x_n\mathbf{a_n} = \mathbf{b}, \]
    a więc $X$ istnieje wtedy i tylko wtedy, gdy kolumna $B$ jest kombinacją liniową kolumn z~$A$~(a więc nie jest liniowo niezależna, ergo $\rank(A) = \rank(\begin{bNiceArray}{cSc}A & B\end{bNiceArray})$).
\end{proof}

\begin{theorem}
    \label{t:rankA = rankA|B = n}
    Układ $AX = B$ ma dokładnie jedno rozwiązanie wtedy i tylko wtedy, gdy
    \[ \rank(A) = \rank(\begin{bNiceArray}{cSc}A & B\end{bNiceArray}) = n, \]
    gdzie $n$ jest liczbą niewiadomych.
\end{theorem}
\begin{proof}
    Jak poprzednio, lecz z wykorzystaniem twierdzenia \ref{t:explicit coefficients}.
\end{proof}

Prosty wniosek z tego twierdzenia jest taki, że jeśli $\rank(A) = \rank(\begin{bNiceArray}{cSc}A & B\end{bNiceArray})$, ale $\rank(A) \neq n$, to układ jest nieoznaczony, a jego rozwiązania zależą od $n - \rank(A)$ parametrów\footnote{jeśli układ jest określony nad ciałem $\RR$ lub $\CC$, to układ nieoznaczony ma nieskończenie wiele rozwiązań}.

Układy równań liniowych można łatwo rozwiązać eliminacją Gaussa w podobny sposób, jak robiliśmy to szukając macierzy odwrotnej w przykładzie \ref{ex:inverse matrix}.

\begin{example}
    Rozwiązać układ równań
    \[ \begin{cases}
        x + 3y - z &= 2 \\
        2x - 3z &= -5 \\
        3x + 2y - 3z &= -1
    \end{cases}. \]
\end{example}
\begin{solution}
    \[ \begin{bNiceArray}{cccIc}
        1 & 3 & -1 & 2 \\
        2 & 0 & -3 & -5 \\
        3 & 2 & -3 & -1
    \end{bNiceArray} \sim \begin{bNiceArray}{cccIc}
        1 & 3 & -1 & 2 \\
        0 & -6 & -1 & -9 \\
        0 & -7 & 0 & -7
    \end{bNiceArray} \sim \begin{bNiceArray}{cccIc}
        1 & 3 & -1 & 2 \\
        0 & 1 & -1 & -2 \\
        0 & 1 & 0 & 1
    \end{bNiceArray} \sim \begin{bNiceArray}{cccIc}
        1 & 3 & -1 & 2 \\
        0 & 1 & -1 & -2 \\
        0 & 0 & 1 & 3
    \end{bNiceArray} \]
    Rząd macierzy jest równy liczbie zmiennych, więc (na mocy twierdzenia \ref{t:rankA = rankA|B = n}) układ jest oznaczony. Teraz możemy kontynuować przekształcenia, aby otrzymać macierz $\begin{bNiceArray}{cSc}I & X\end{bNiceArray}$, ale w praktyce łatwiej bedzie teraz wrócic do układu równań. Mamy więc
    \[ z = 3, \]
    \[ y - z = -2 \implies y = -2 + 3 = 1, \]
    \[ x + 3x - z = 2 \implies x = 2 - 3 + 3 = 2. \]
\end{solution}