\begin{definition}
    Szereg liczbowy to para $((a_n)_{n\in\NN}, (S_n)_{n\in\NN})$, gdzie $S_n = \sum_{i=1}^n a_i$.
\end{definition}

Mówimy, że szereg liczbowy jest \vocab{zbieżny}, jeśli istnieje skończona granica $\lim\limits_{n\lthen\infty} S_n = S$. Liczbę $S$ nazywamy wtedy \vocab{sumą} tego szeregu.

\begin{theorem}[warunek konieczny zbieżności szeregu]
    \label{t:necessary condition of convergence}
    Jeśli szereg
    \[ \sum_{n = 1}^\infty a_n \]
    jest zbieżny, to
    \[ \lim_{n\lthen\infty} a_n = 0. \]
\end{theorem}

\begin{example}
    Znajdź sumę szeregu
    \[ \sum_{n=1}^\infty \frac{1}{n(n+2)}. \]
\end{example}
\begin{solution}
    Wykorzystamy tak zwane \vocab{sumy teleskopowe}.
    \begin{align*}
        &\sum_{n=1}^\infty \frac{1}{n(n+2)} = \frac{1}{2}\sum_{n=1}^\infty \left(\frac{1}{n} - \frac{1}{n+2}\right) \\
        =&\ \frac{1}{2}\lim_{n\lthen\infty}\left(\frac{1}{1} - \frac{1}{3} + \frac{1}{2} - \frac{1}{4} + \frac{1}{3} - \frac{1}{5} + \ldots + \frac{1}{n} - \frac{1}{n+2}\right) \\
        =&\ \frac{1}{2}\lim_{n\lthen\infty}\left(1 + \frac{1}{2} - \frac{1}{n+1} - \frac{1}{n+2}\right) = \frac{3}{4}
    \end{align*}
\end{solution}

Można łatwo pokazać, że szereg harmoniczny $\sum_{n=1}^\infty \frac{1}{n}$ nie jest zbieżny (czyli jest \vocab{rozbieżny}), mimo że spełnia warunek konieczny:
\[ \underbrace{\left(\frac{1}{1}\right)}_1 + \underbrace{\left(\frac{1}{2} + \frac{1}{3}\right)}_{>1} + \underbrace{\left(\frac{1}{4} + \frac{1}{5} + \frac{1}{6} + \frac{1}{7}\right)}_{>1} + \ldots .\]

Okazuje się, że zachodzi również dużo mocniejsze twierdzenie:
\begin{theorem}[o zbieżności szeregów harmonicznych]
    \label{t:harmonic series}
    Szereg harmoniczny rzędu $\alpha \in \RR$
    \[ \sum_{n=1}^\infty \frac{1}{n^\alpha} \]
    jest zbieżny wtedy i tylko wtedy, gdy $\alpha > 1$.
\end{theorem}

Jeśli szereg $\sum_{n=1}^\infty |a_n|$ jest zbieżny, to mówimy, że szereg $\sum_{n=1}^\infty a_n$ jest \vocab{bezwzględnie zbieżny}, w przeciwnym przypadku jest \vocab{warunkowo zbieżny}. Bezwzględna zbieżność szeregu pociąga za sobą jego zbieżność.

Aby sprawdzić zbieżność szeregów stosuje się kilka kryteriów zbieżności.

\begin{theorem}[kryterium porównawcze]
    Jeśli dla każdego $n$ wiekszego od pewnego $n_0$ zachodzi
    \[ a_n \leq b_n \]
    oraz $a_n, b_n > 0$, to ze zbieżności szeregu $\sum_{n=1}^\infty b_n$ wynika zbieżność $\sum_{n=1}^\infty a_n$, a z rozbieżności szeregu $\sum_{n=1}^\infty a_n$ wynika rozbieżność $\sum_{n=1}^\infty b_n$.
\end{theorem}

\begin{remark*}
    \begin{multicols}{2}
    Wraz z powyższym twierdzeniem warto stosować nierówności, które zachodzą w przedziale $[0,1]$:
    \begin{itemize}
        \item $\frac{x}{2} \leq {\color{MainColor1} \sin{x}} \leq x$
        \item $\frac{x}{2} \leq {\color{BoxColor2} \ln(x + 1)} \leq x$
        \item $x \leq {\color{LinkColor2} \tan{x}} \leq 2x$
        \item $1 - x \leq \cos{x}$
    \end{itemize}
    \hspace{2em}

    \begin{tikzpicture}[scale=1]
        \begin{axis}[domain=0:1, restrict y to domain=0:1.5,
            xmin=0, xmax=1, ymin=0, ymax=1, grid=both,
            xtick distance=0.25, ytick distance=0.25,
            tick label style={font=\scriptsize}, xscale=0.78, yscale=0.78]
            \addplot[samples=10,smooth,ultra thin] {x/2};
            \addplot[samples=10,smooth,ultra thin] {x};
            \addplot[samples=10,smooth,ultra thin] {2*x};
            \addplot[samples=100,smooth,very thick,MainColor1] {sin(deg(x))};
            \addplot[samples=100,smooth,very thick,LinkColor2] {tan(deg(x))};
            \addplot[samples=100,smooth,very thick,BoxColor2] {ln(x + 1)};
        \end{axis}
    \end{tikzpicture}
    \end{multicols}
\end{remark*}

\begin{example}
    Zbadaj zbieżność szeregu
    \[ \sum_{n=1}^\infty \ln\left(\frac{n^2 + 1}{n^2}\right). \]
\end{example}
\begin{solution}
    \[ \sum_{n=1}^\infty \ln\left(\frac{n^2 + 1}{n^2}\right) = \sum_{n=1}^\infty \ln\left(1 + \frac{1}{n^2}\right) \]
    Wyrazy szeregu są dodatnie oraz dla każdego $n \in \NN$
    \[ \ln\left(1 + \frac{1}{n^2}\right)  < \frac{1}{n^2}, \]
    więc, na podstawie twierdzenia \ref{t:harmonic series}, dany szereg jest zbieżny.
\end{solution}

\begin{theorem}[kryterium ilorazowe]
    Jeśli dla każdego $n$ wiekszego od pewnego $n_0$ wyrazy szeregów $\sum_{n=1}^\infty a_n$ i $\sum_{n=1}^\infty b_n$ są dodatnie oraz
    \[ \lim_{n\lthen\infty} \frac{a_n}{b_n} = g \in (0, \infty), \]
    to dane szeregi są jednocześnie zbieżne lub jednocześnie rozbieżne.
\end{theorem}

\begin{theorem}[kryterium d'Alemberta]
    Niech będzie dany szereg $\sum_{n=1}^\infty a_n$ o niezerowych wyrazach oraz niech
    \[ \lim_{n\lthen\infty} \frac{a_{n+1}}{a_n} = g. \]
    Jeśli $g > 1$, to dany szereg jest rozbieżny, a jeśli $g < 1$, to szereg jest zbieżny.
\end{theorem}

\begin{theorem}[kryterium Cauchy'ego]
    Niech będzie dany szereg $\sum_{n=1}^\infty a_n$ oraz niech
    \[ \lim_{n\lthen\infty} \sqrt[n]{|a_n|} = g. \]
    Jeśli $g > 1$, to dany szereg jest rozbieżny, a jeśli $g < 1$, to szereg jest zbieżny.
\end{theorem}

\begin{remark*}
    Jeśli w kryteriach d'Alemberta lub Cauchy'ego wyjdzie $g = 1$, to nie możemy powiedzieć nic o zbieżności ciągu.
\end{remark*}

\begin{example}
    Zbadaj zbieżność szeregu
    \[ \sum_{n=1}^\infty \frac{3^n \cdot n}{4^n}. \]
\end{example}
\begin{solution}
    Korzystając z kryterium Cauchy'ego mamy
    \[ \lim_{n\lthen\infty} \sqrt[n]{\frac{3^n \cdot n}{4^n}} = \lim_{n\lthen\infty} \frac{3}{4} \cdot \sqrt[n]{n} = \frac{3}{4} < 1, \]
    więc dany szereg jest zbieżny.
\end{solution}

\begin{theorem}[kryterium całkowe]
    Jeśli dla każdego $n$ wiekszego od pewnego $n_0$ wyrazy szeregu $\sum_{n=1}^\infty a_n$ są dodatnie oraz istnieje taka malejąca (na przedziale $[n_0, \infty)$) funkcja $f$, że $a_n = f(n)$ dla każdego $n$, to szereg
    \[ \sum_{n=1}^\infty a_n \]
    jest zbieżny wtedy i tylko wtedy, gdy całka niewłaściwa
    \[ \int_1^\infty f(x) \d x \]
    jest zbieżna.
\end{theorem}

\begin{theorem}[kryterium Leibniza]
    Dany jest szereg $\sum_{n=1}^\infty (-1)^na_n$. Jeśli ciąg $(a_n)$ jest dodatni, zbieżny do zera oraz malejący, to jest dany szereg jest zbieżny.
\end{theorem}

Szereg opisywany przez kryterium Leibniza nazywamy szeregiem \vocab{naprzemiennym}.

\begin{example}
    Zbadać zbieżność warunkową i bezwzględną szeregu
    \[ \sum_{n=1}^\infty (-1)^n \frac{1}{n\ln{n}}. \]
\end{example}
\begin{solution}
    Korzystając z kryterium Leibniza bardzo łatwo pokazać, że dany szereg jest zbieżny. Ciąg $a_n = \frac{1}{n\ln{n}}$ ma oczywiście wyrazy dodatnie i jest zbieżny do zera. Ponadto jest malejący, bo zarówno $n$, jak i $\ln{n}$ rosną.

    Aby określić, czy dany szereg jest bezwzględnie zbieżny skorzystamy z kryterium całkowego.
    \[ \int \frac{1}{x\ln{x}} \d x = \left|\begin{alignedat}{2}&u = \ln{x} \\ &\!\d u = \frac{1}{x} \d x\end{alignedat}\right| = \int \frac{1}{u} \d u = \ln{u} + C = \ln(\ln(x)) + C. \]
    \[ \int_1^\infty \frac{1}{x\ln{x}} \d x = \ln(\ln(x)) \Big|_1^\infty \text{ -- rozbieżna.} \]
    Z tego wynika, że dany szereg jest tylko warunkowo zbieżny.
\end{solution}