W tej sekcji będziemy skupiać się na funkcjach typu $\RR^k \to \RR$. W tym kontekście warto zauważyć, że struktura $(\RR^k, \RR, +, \cdot)$ jest przestrzenią wektorową. Jest ona również przestrzenią Banacha ze zdefiniowaną normą euklidesową.

\begin{fact}[granica ciągu]
    Weźmy ciąg $(x_n)$ elementów zbioru $\RR^k$ i oznaczmy $x_n = (x_{n,1}, x_{n,2}, \ldots, x_{n,k})$. Zachodzi równoważność
    \[ \lim_{n\to\infty} x_n = (g_1, g_2, \ldots, g_k) \iff \dforall{1\leq i\leq k} \lim_{n\to\infty}x_{n,i} = g_i. \]
\end{fact}

\begin{definition}[Heinego]
    Funkcja $f : D \to \RR$, gdzie $D\subset\RR^k$, ma granicę w punkcie $x_0$ równą $g$ wtedy i tylko wtedy, gdy dla każdego ciągu $(x_n)$ takiego, że $x_n \in D, x_n \neq x_0$ oraz $\lim_{n\to\infty} x_n = x_0$ zachodzi
    \[ \lim_{n\to\infty} f(x_n) = g. \]
\end{definition}

\begin{example}
    Zbadaj granicę
    \[ \lim_{(x, y)\to (0, 0)} \frac{x^2}{x^2 + y^2}. \]
\end{example}
\begin{solution}
    Podstawiając $x = 0$ mamy
    \[ \lim_{y\to 0} \frac{0}{0 + y^2} = 0, \]
    a dla $y = 0$ otrzymujemy
    \[ \lim_{x\to 0} \frac{x^2}{x^2 + 0} = 1, \]
    więc granica nie istnieje. Bardziej formalnie możemy powiedzieć, że wzięliśmy dwa ciągi: $a_n = (0, \frac{1}{n}), b_n = (\frac{1}{n}, 0)$ i pokazaliśmy sprzeczność z definicją Heinego.
\end{solution}

\begin{figure}[H]
    \centering
    \begin{tikzpicture}[scale=0.8]
        \begin{axis}[my axis style, view = {20}{30}]
            \addplot3[
                domain = -2:2, domain y = -2:2,
                samples = 30, samples y = 30,
                surf, opacity=.4
            ]{x^2/(x^2 + y^2)};
        \end{axis}
    \end{tikzpicture}
    \caption{Wykres funkcji $f(x, y) = \frac{x^2}{x^2 + y^2}$.}
\end{figure}

\begin{example}
    Zbadaj granicę
    \[ \lim_{(x, y)\to (0, 0)} \frac{\sin(x^2 + y^2)}{x^2 + y^2}. \]
\end{example}
\begin{solution}
    \begin{align*}
        &\lim_{(x, y)\to (0, 0)} \frac{\sin(x^2 + y^2)}{x^2 + y^2} = \left|\begin{alignedat}{2}&x = r\cos\varphi \\ &y = r\sin\varphi\end{alignedat}\right| = \lim_{r\to 0} \frac{\sin(r^2\cos^2\varphi + r^2\sin^2\varphi)}{r^2\cos^2\varphi + r^2\sin^2\varphi} = \\
        &= \lim_{r\to 0} \frac{\sin(r^2)}{r^2} = \lim_{t\to 0} \frac{\sin t}{t} = 1.
    \end{align*}
\end{solution}

\begin{figure}[H]
    \centering
    \begin{tikzpicture}[scale=0.8]
        \begin{axis}[my axis style, view = {20}{30}]
            \addplot3[
                domain = -2:2, domain y = -2:2,
                samples = 30, samples y = 30,
                surf, opacity=.4
            ]{sin(deg(x^2 + y^2))/(x^2 + y^2)};
        \end{axis}
    \end{tikzpicture}
    \caption{Wykres funkcji $f(x, y) = \frac{\sin(x^2 + y^2)}{x^2 + y^2}$.}
\end{figure}

\begin{example}
    Zbadaj granicę
    \[ \lim_{(x, y)\to (0, 0)} \frac{xy^2}{x^2 + y^2}. \]
\end{example}
\begin{solution}
    Skoro
    \[ 0 \leq \left|\frac{xy^2}{x^2 + y^2}\right| = |x|\frac{y^2}{x^2 + y^2} \leq |x| \]
    oraz $\lim_{(x, y)\to (0, 0)} 0 = \lim_{(x, y)\to (0, 0)} |x| = 0$, to, na mocy twierdzenia o trzech ciągach,
    \[ \lim_{(x, y)\to (0, 0)} \left|\frac{xy^2}{x^2 + y^2}\right| = 0, \]
    więc
    \[ \lim_{(x, y)\to (0, 0)} \frac{xy^2}{x^2 + y^2} = 0. \]
\end{solution}

\begin{figure}[H]
    \centering
    \begin{tikzpicture}[scale=0.8]
        \begin{axis}[my axis style, view = {-15}{30}]
            \addplot3[
                domain = -2:2, domain y = -2:2,
                samples = 30, samples y = 30,
                surf, opacity=.4
            ]{(x * y^2)/(x^2 + y^2)};
        \end{axis}
    \end{tikzpicture}
    \caption{Wykres funkcji $f(x, y) = \frac{xy^2}{x^2 + y^2}$.}
\end{figure}

\begin{example}
    Zbadaj granicę
    \[ \lim_{(x, y)\to (0, 0)} \frac{xy^2}{x^2 + y^4} \]
\end{example}
\begin{solution}
    Podstawiając $y = x$ mamy
    \[ \lim_{x\to 0} \frac{x^3}{x^2 + x^4} = 0, \]
    a dla $x = y^2$ otrzymujemy
    \[ \lim_{y\to 0} \frac{y^4}{y^4 + y^4} = \frac{1}{2}, \]
    więc granica nie istnieje.
\end{solution}

\begin{remark*}
    Powyższy przykład jest o tyle ciekawy, że jeśli $x$ oraz $y$ zbiegają w tym samym tempie (czyli łączy jest liniowa zależność) to zawsze granica wyjdzie zerowa. Aby pokazać ten fakt, przejdziemy do współrzędnych biegunowych:
    \[ \lim_{(x, y)\to (0, 0)} \frac{xy^2}{x^2 + y^4}  = \lim_{r\to 0} \frac{r^3 \cos\varphi\sin^2\varphi}{r^2\cos^2\varphi + r^4\sin^4\varphi} = \lim_{r\to 0} \frac{r \cos\varphi\sin^2\varphi}{\cos^2\varphi + r^2\sin^4\varphi}. \]
    Jeśli $\varphi = \pm\frac{\pi}{2}$, to (skoro $\cos\varphi = 0$)
    \[ \lim_{r\to 0} \frac{r \cos\varphi\sin^2\varphi}{\cos^2\varphi + r^2\sin^4\varphi} = \lim_{r\to 0} \frac{0}{0 \pm r^2} = 0, \]
    a jeśli $\varphi \neq \pm\frac{\pi}{2}$, to (skoro $\sin$ i $\cos$ są ograniczone)
    \[ \lim_{r\to 0} \frac{r \cos\varphi\sin^2\varphi}{\cos^2\varphi + r^2\sin^4\varphi} = \frac{0}{\cos^2\varphi + 0} = 0. \]

    Natomiast jeśli $\varphi$ nie jest stałe, ale na przykład zbiega do $\frac{\pi}{2}$, to, jak można zauważyć na poniższym rysunku, granica niekoniecznie będzie zerowa.
\end{remark*}

\begin{figure}[H]
    \centering
    \begin{tikzpicture}[scale=0.8]
        \begin{axis}[my axis style, view = {-10}{40}]
            \addplot3[
                thick, AccColor4, samples=10,
                domain=-1:1,
            ]({x},{x/3},{x*(x/3)^2/(x^2 + (x/3)^4)});
            \addplot3[
                domain = -1:1, domain y = -1:1,
                samples = 30, samples y = 30,
                surf, opacity=.4
            ]{x*y^2/(x^2 + y^4)};
        \end{axis}
    \end{tikzpicture}
    \begin{tikzpicture}[scale=0.8]
        \begin{axis}[my axis style, view = {0}{0}]
            \addplot3[
                domain = -1:1, domain y = -1:1,
                samples = 30, samples y = 30,
                surf, opacity=.4
            ]{x*y^2/(x^2 + y^4)};
            \addplot3[
                thick, AccColor4, samples=10,
                domain=-1:1,
            ]({x},{x/3},{x*(x/3)^2/(x^2 + (x/3)^4)});
        \end{axis}
    \end{tikzpicture}
    \caption[short]{Wykres funkcji $f(x, y) = \frac{xy^2}{x^2 + y^4}$ z zaznaczoną prostą $y = \frac{x}{3}$.}
\end{figure}

Granicę funkcji $f(x, y)$ w formie $\lim_{(x, y) \to (x_0, y_0)} f(x, y)$ nazywamy czasami \vocab{granicą podwójną}\footnote{co może być nazwą mylącą; w literaturze angielskiej jest to \textit{ordinary limit}, który nie jest tym samym pojęciem co \textit{double limit}. W szczególności dla \textit{double limit} nie zachodzi fakt \ref{f:iterated and ordinary limit}, zobacz też: \href{https://en.wikipedia.org/wiki/Iterated_limit\#Limit_of_function}{wikipedia}.}, w odróżnieniu od granic $\lim_{x\to x_0} \lim_{y\to y_0} f(x, y)$ czy $\lim_{y\to y_0} \lim_{x\to x_0} f(x, y)$, które są \vocab{granicami iterowanymi}.

\begin{fact}
    \label{f:iterated and ordinary limit}
    Jeśli funkcja $f(x, y)$ ma w punkcie $(x_0, y_0)$ granicę podwójną oraz istnieją obie jej granice iterowane, to wszystkie trzy są sobie równe.
\end{fact}
\begin{proof}[Uzasadnienie]
    Granica iterowana w punkcie $(x_0, y_0)$ modeluje dążenie do tego punktu po dwóch bokach prostokąta.
\end{proof}

\begin{remark*}
    Jeśli obie granice iterowane nie istnieją, to nie znaczy, że granica podwójna nie istnieje. Jeśli obie granice iterowanie istnieją i są sobie równe, to nie znaczy, że granica podwójna istnieje.

    Natomiast z powyższego faktu wynika, że jeśli obie granice iterowane istnieją i nie są sobie równe, to granica podwójna nie istnieje.
\end{remark*}

\begin{fact}
    Jeśli funkcja (wielu zmiennych) $f$ jest ciągła w $x_0$, a funkcja $g$ jest ciągła w $f(x_0)$, to funkcja $g\circ f$ jest ciągła w $x_0$.
\end{fact}

\begin{fact}
    Jeśli funkcje (wielu zmiennych) $f, g$ są ciągłe w $x_0$, to funkcje $f+g, f-g, f\cdot g$ również są ciągłe w tym punkcie. Jeśli dodamy warunek, że $g(x) \neq 0$ dla pewnego otoczenia $x_0$, to ciągła w tym puncie jest również funkcja $\frac{f}{g}$.
\end{fact}