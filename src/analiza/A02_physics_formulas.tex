Zakładamy, że $\rho$ jest funkcją gęstości $D \to \RR_+$ lub $V \to \RR_+$, gdzie $D$ jest obszarem regularnym w $\RR^2$, a $V$ w $\RR^3$.

\subsection*{Całki podwójne}

\paragraph*{Masa obszaru $D$}
\[ M = \iint\limits_D \rho(x, y) \d x \d y \]

\paragraph*{Moment statyczny obszaru $D$ względem osi $OX$}
\[ M_{OX} = \iint\limits_D y\rho(x, y) \d x \d y \]

\paragraph*{Moment statyczny obszaru $D$ względem osi $OY$}
\[ M_{OY} = \iint\limits_D x\rho(x, y) \d x \d y \]

\paragraph*{Środek ciężkości obszaru $D$}
\[ x_0 = \frac{M_{OY}}{M}, \qquad y_0 = \frac{M_{OX}}{M} \]

\paragraph*{Moment bezwładności obszaru $D$ względem środka układu współrzędnych}
\[ M_B = \iint\limits_D (x^2 + y^2) \rho(x, y) \d x \d y \]

\subsection*{Całki potrójne}

\paragraph*{Masa bryły $V$}
\[ M = \iiint\limits_V \rho(x, y, z) \d x \d y \d z \]

\paragraph*{Momenty statyczne bryły $V$ względem płaszczyzn}
\[ M_{OXY} = \iiint\limits_V z\rho(x, y, z) \d x \d y \d z \]
\[ M_{OXZ} = \iiint\limits_V y\rho(x, y, z) \d x \d y \d z \]
\[ M_{OYZ} = \iiint\limits_V x\rho(x, y, z) \d x \d y \d z \]

\paragraph*{Środek ciężkości bryły $V$}
\[ x_0 = \frac{M_{OYZ}}{M}, \qquad y_0 = \frac{M_{OXZ}}{M}, \qquad z_0 = \frac{M_{OXY}}{M} \]

\paragraph*{Moment bezwładności bryły $V$ względem osi $OZ$}
\[ M_B = \iiint\limits_V (x^2 + y^2) \rho(x, y, z) \d x \d y \d y \]