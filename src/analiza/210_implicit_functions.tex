\begin{definition}
    Funkcja uwikłana określona przez równanie $F(x, y) = 0$ to każda funkcja $y(x)$ spełniająca równość
    \[ F(x, y(x)) = 0 \]
    dla wszystkich $x$ w otoczeniu pewnego punktu $x_0$. Jeśli taka funkcja istnieje, to mówimy, że równanie $F(x, y) = 0$ możemy rozwiłać w otoczeniu tego punktu.
\end{definition}

\begin{theorem}[o funkcji uwikłanej]
    Jeśli funkcja $F : \RR^2 \supset D \to \RR$ jest różniczkowalna w sposób ciągły w otoczeniu punktu $(x_0, y_0)$ i $F(x_0, y_0) = 0$ oraz $\frac{\p F}{\p y}(x_0, y_0) \neq 0$, to istnieje jednoznacznie określona funkcja uwikłana $y = y(x)$. Ponadto
    \[ y'(x_0) = -\frac{\frac{\p F}{\p x}(x_0, y_0)}{\frac{\p F}{\p y}(x_0, y_0)} \]
    oraz, jeśli $y'(x_0) = 0$,
    \[ y''(x_0) = -\frac{\frac{\p{}^2 F}{\p x^2}(x_0, y_0)}{\frac{\p{}^2 F}{\p y^2}(x_0, y_0)}. \]
\end{theorem}