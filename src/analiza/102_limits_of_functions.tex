\vocab{Otoczeniem} $U(x_0, r)$ punktu $x_0 \in \RR$ o promieniu $r > 0$ nazywamy przedział $(x_0 - r, x_0 + r)$, a jego \vocab{sąsiedztwem} $S(x_0, r)$ -- otoczenie bez niego samego, czyli $U(x_0, r) \setminus \{x_0\}$. Definiujemy również sąsiedztwo lewo- i prawostronne punktu $x_0$ -- odpowiednio zbiory $S^-(x_0, r) = (x_0 - r, x_0)$ i $S^+(x_0, r) = (x_0, x_0 + r)$. Dla $\infty$ każde sąsiedztwo jest sąsiedztwem lewostronnym, a dla $-\infty$ prawostronnym.

\begin{definition}[Heinego granicy funkcji]
    Funkcja $f: \RR \supset D_f \to \RR$ ma granicę $g = \lim\limits_{x \to x_0} f(x)$ w $x_0 \in \ol{\RR}$ wtedy, gdy jest określona w sąsiedztwie punktu $x_0$ oraz dla każdego ciągu $(x_n)$ takiego, że $\forall n \in \NN : x_n \neq x_0, x_n \in D_f$ oraz $(x_n) \to x_0$ zachodzi
    \[ \lim_{n\to\infty} f(x_n) = g. \]
\end{definition}

Granice lewo- lub prawostronne są definiowane analogicznie, lecz funkcja $f$ musi być zdefiniowana w lewo- lub prawostronnym sąsiedztwie punktu $x_0$, a elementy ciągu $x_n$ muszą leżeć po lewej lub prawej stronie od $x_0$.

\begin{theorem}
    Funkcja $f$ ma granicę wtedy i tylko wtedy, gdy obie granice jednostronne istnieją i są sobie równe.
    \[ \lim_{x \to x_0} f(x) = g \quad\iff\quad \lim_{x \to x_0^-} f(x) = g \land \lim_{x \to x_0^+} f(x) = g \]
\end{theorem}
\begin{proof}
    Wynika wprost z definicji Heinego granicy funkcji i granic jednostronnych.
\end{proof}

\begin{definition}[Cauchy'ego granicy funkcji]
    \label{d:cauchy_lim}
    Funkcja $f: \RR \supset D_f \to \RR$ ma granicę $g = \lim\limits_{x \to x_0} f(x)$ w $x_0 \in \ol{\RR}$ wtedy, gdy jest określona w sąsiedztwie punktu $x_0$ oraz zachodzi warunek
    \[ \dforall{\eps > 0} \dexists{\delta > 0} \dforall{x \in S(x_0, \delta)} |f(x) - g| < \eps. \]
\end{definition}

\begin{theorem}[o arytmetyce granic funkcji]
    Jeśli funkcje $f$ i $g$ są określone w sąsiedztwie $x_0 \in \ol{\RR}$, to:
    \begin{enumerate}
        \item $\lim\limits_{x\to x_0} (f(x) \pm g(x)) = \lim\limits_{x\to x_0} f(x) \pm \lim\limits_{x\to x_0} g(x),$
        \item $\lim\limits_{x\to x_0} (f(x) \cdot g(x)) = \lim\limits_{x\to x_0} f(x) \cdot \lim\limits_{x\to x_0} g(x),$
        \item $\lim\limits_{x\to x_0} \frac{f(x)}{g(x)} = \frac{\lim\limits_{x\to x_0} f(x)}{\lim\limits_{x\to x_0} g(x)},$ jeśli $g(x) \neq 0$ w sąsiedztwie $x_0$ oraz $\lim\limits_{x\to x_0} g(x) \neq 0$.
    \end{enumerate}
\end{theorem}
\begin{proof}
    Wynika w prosty sposób z definicji $\ref{d:cauchy_lim}$.
\end{proof}

\begin{theorem}[o trzech funkcjach]
    \label{t:squeeze theorem}
    Jeśli $\lim\limits_{x\to x_0} f(x) = \lim\limits_{x\to x_0} h(x) = g$ oraz dla każdego $x$ w sąsiedztwie $x_0$ zachodzi
    \[ f(x) \leq g(x) \leq h(x), \]
    to
    \[ \lim\limits_{x\to x_0} g(x) = g. \]
\end{theorem}
\begin{proof}
    Analogiczny jak dowód twierdzenia o trzech ciągach (\ref{t:sequence squeeze theorem}).
\end{proof}

\begin{remark}
    Oprócz arytmetyki granicy czy twierdzenia o trzech funkcjach, prawdziwych jest również kilka innych twierdzenia, które udowodnialiśmy dla ciągów, między innymi o funkcji ograniczonej i zbieżnej do $0$ (\ref{t:sequence bounded and convergent to 0}) czy granice specjalnych funkcji (\ref{l:lim e^k}).
\end{remark}

\begin{example}  % source: Demidovich, p. 26
    Znajdź
    \[ \lim_{x \to 0} \frac{\sqrt{1+ x} - 1}{\sqrt[3]{1 + x} - 1}. \]
\end{example}
\begin{solution}
    Biorąc
    \[ 1 + x = y^6, \]
    mamy
    \[ \lim_{x \to 0} \frac{\sqrt{1+ x} - 1}{\sqrt[3]{1 + x} - 1} = \lim_{y \to 1} \frac{y^3 - 1}{y^2 - 1} = \lim_{y \to 1} \frac{y^2 + y + 1}{y + 1} = \frac{3}{2}. \]
\end{solution}

\begin{theorem}
    \label{t:lim_sinx/x}
    Zachodzi równość
    \[ \lim_{x \to 0}\frac{\sin{x}}{x} = 1. \]
\end{theorem}
\begin{proof}
    Narysujmy pewne długości na okręgu jednostkowym i oznaczmy jak na rysunku.
    \begin{center}
        \begin{tikzpicture}[scale=2.8]
            \tkzInit[xmin=-.3, ymin=-.3, xmax=1.5, ymax=1.2]
            \tkzClip
            \tkzSetUpLabel[font=\scriptsize]
            \tkzDefPoints{0/0/A,1/0/D,1/1/D'}
            \tkzDefPoint(38:1){C}
            \tkzDefPointBy[projection=onto A--D](C) \tkzGetPoint{B}
            \tkzInterLL(A,C)(D,D') \tkzGetPoint{E}
            \tkzDrawCircle(A,D)
            \tkzDrawSegments(A,E A,D B,C D,E C,D)
            \tkzMarkAngle[size=4mm](D,A,C)
            \tkzLabelAngle[pos=.28](D,A,C){$x$}
            \tkzDrawPoints(A,B,C,D,E)
            \tkzLabelPoints[below](A,B,D)
            \tkzLabelPoints[above](C,E)
        \end{tikzpicture}
    \end{center}
    Jeśli $\angle DAC = x$ oraz $|AC| = 1$, to $|BC| = |\sin x|$ i $|DE| = |\tan x|$. Między polem trójkąta $\triangle ADC$, polem wycinka koła $A\arc{DC}$ i polem trójkąta $\triangle ADE$ zachodzi poniższa nierówność
    \[ [ADC] \leq [A\arc{DC}] \leq [ADE], \]
    a więc
    \[ \frac{|\sin x|}{2} \leq \frac{|x|}{2} \leq \frac{|\tan x|}{2} \]
    \[ |\sin x| \leq |x| \leq |\tan x| \]
    \[ 1 \leq \frac{|x|}{|\sin{x}|} \leq \frac{1}{|\cos x|}. \]
    Przy $x$ bliskim $0$ możemy zapisać
    \[ 1 \leq \frac{x}{\sin{x}} \leq \frac{1}{\cos x} \]
    \[ 1 \geq \frac{\sin{x}}{x} \geq \cos x. \]
    Z twierdzenia o trzech funkcjach (\ref{t:squeeze theorem}) otrzymujemy
    \[ \lim_{x \to 0} \frac{\sin{x}}{x} = 1. \]
\end{proof}

\begin{example}
    Oblicz
    \[ \lim_{x \to 0} \frac{\cos{3x} - \cos{2x}}{x^2}. \]
\end{example}
\begin{solution}
    Korzystając ze wzoru na różnicę cosinusów mamy
    \[ \lim_{x \to 0} \frac{\cos{3x} - \cos{2x}}{x^2} = \lim_{x \to 0} -2\frac{\sin\frac{5x}{2}\sin\frac{x}{2}}{x^2}. \]
    Na mocy twierdzenia \ref{t:lim_sinx/x} otrzymujemy
    \[ \lim_{x \to 0} -2\frac{\sin\frac{5x}{2}\sin\frac{x}{2}}{x^2} = -2\cdot\frac{\frac{5}{2}\cdot\frac{1}{2}}{1}\lim_{x\to 0}\frac{\sin\frac{5x}{2}}{\frac{5x}{2}}\lim_{x\to 0}\frac{\sin\frac{x}{2}}{\frac{x}{2}} =  \frac{-5}{2}. \]
\end{solution}

\begin{theorem}
    \label{t:lim ln(1+x)/x}
    Zachodzi równość
    \[ \lim_{x\to 0}\frac{\ln(1 + x)}{x} = 1. \]
\end{theorem}
\begin{proof}
    TODO (logarytm granicy = granica logarytmu)
    \[ \lim_{x\to 0}\frac{\ln(1 + x)}{x} = \lim_{x\to 0} \ln((1 + x)^{\frac{1}{x}}) = \ln\left(\lim_{x\to 0} (1 + x)^{\frac{1}{x}}\right) = \ln e = 1 \]
\end{proof}

\begin{theorem}
    Zachodzi równość
    \[ \lim_{x\to 0}\frac{a^x - 1}{x} = \ln a \]
    dla $a > 0$.
\end{theorem}
\begin{proof}
    Skorzystamy z twierdzenia \ref{t:lim ln(1+x)/x}. Podstawiając $y = a^x - 1$ mamy
    \begin{align*} \lim_{x\to 0}\frac{a^x - 1}{x} &= \lim_{y\to 0}\frac{y}{\log_a(1 + y)} = \lim_{y\to 0}\frac{y\ln{a}}{\ln {a}\log_a(1 + y)} = \\
    &= \lim_{y\to 0}\frac{y\ln{a}}{\ln(1 + y)} = \lim_{y\to 0}\frac{y\ln{a}}{y} = \ln{a}. \end{align*}
\end{proof}

\begin{theorem}
    Jeśli $\lim\limits_{x\to x_0} f(x) = \pm\infty$, to zachodzi równość
    \[ \lim_{x\to x_0}\left(1 + \frac{1}{f(x)}\right)^{f(x)} = e. \]
\end{theorem}
\begin{proof}
    TODO
\end{proof}

\begin{theorem}[o granicy funkcji złożonej]
    Jeśli $\lim\limits_{x\to x_0} f(x) = y_0, \lim\limits_{x\to y_0} g(x) = z_0$ oraz dla każdego punktu $x$ w sąsiedztwie $x_0$ $f(x) \neq y_0$, to
    \[ \lim_{x\to x_0} g(f(x)) = z_0. \]
\end{theorem}
\begin{proof}
    TODO
\end{proof}

\subsection{Ciągłość funkcji}
\begin{definition}[ciągłość funkcji w punkcie]
    Jeśli funkcja $f$ jest określona w otoczeniu punktu $x_0 \in D_f$ to mówimy, że funkcja $f$ jest ciągła w tym punkcie, jeśli
    \[ \lim_{x \to x_0} f(x) = f(x_0). \]
\end{definition}

Mówimy, że funkcja jest \vocab{ciągła}, jeśli jest ciągła w każdym punkcie swojej dziedziny.

\begin{fact}
    Suma, różnica, iloczyn oraz iloraz (o ile mianownik się nie zeruje) funkcji jest funkcją ciągłą.
\end{fact}
\begin{proof}
    Wynika z arytmetyki granic funkcji.
\end{proof}

\begin{fact}
    Wszystkie funkcje elementarne (funkcje wielomianowe, wymierne i niewymierne, logarytmiczne, trygonometryczne, cyklometryczne oraz wszystkie ich złożenia) są ciągłe w swojej dziedzinie.
\end{fact}
\begin{proof}
    Wystarczy wykazać ciągłość funkcji: identyczności, stałej, sinus, arcus sinus oraz logarytmu i skorzystać z poprzedniego faktu.
\end{proof}

\begin{theorem}[o lokalnym zachowaniu znaku]
    Jeśli funkcja $f$ jest ciągła w $x_0$ oraz $f(x_0) \neq 0$, to istnieje takie otoczenie $U(x_0, r)$, że dla każdego $x \in U(x_0, r)$ wartość $f(x)$ jest tego samego znaku co $f(x_0)$.
\end{theorem}
\begin{proof}
    Z definicji Cauchy'ego (\ref{d:cauchy_lim}).
\end{proof}

\begin{theorem}[Darboux, o wartości pośredniej]
    Każda ciągła funkcja $f$ ma własność Darboux, to znaczy, że jeśli $f(a)f(b) < 0$, to istnieje takie $c \in (a, b)$, że
    \[ f(c) = 0. \]
\end{theorem}

\begin{theorem}[Weierstrassa, o osiąganiu kresów]
    \label{t:Weierstrass}
    Każda funkcja $f$ ciągła na przedziale domkniętym $[a, b]$ ma wartość najmniejszą oraz wartość największą na tym przedziale.
\end{theorem}
