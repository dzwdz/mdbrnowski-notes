Całki wielokrotne definiujemy podobnie jak całki funkcji jednej zmiennej --- dzielimy dziedzinę na $n$ małych prostokątów (lub sześcianów, hipersześcianów) i sprawdzamy, czy istnieje granica przy $n \to \infty$ sumy iloczynów ich pól (objętości) i wartości funkcji w pewnych ich punktach.

\subsection{Całka podwójna}
W tej sekcji zajmiemy się całkami funkcji dwóch zmiennych $f : P \to \RR$, gdzie $P \supset \RR^2$.

\begin{theorem}[warunek wystarczający całkowalności funkcji po prostokącie]
    Jeśli funkcja $f$ jest ciągła na prostokącie $P$, to jest na nim całkowalna.
\end{theorem}

\begin{theorem}[Fubiniego]
    Jeśli funkcja $f$ jest całkowalna na prostokącie $P = [a_1, b_1] \times [a_2, b_2]$, to
    \[ \iint\limits_P f(x, y) \d x \d y = \int\limits_{a_1}^{b_1}\left(\int\limits_{a_2}^{b_2} f(x, y) \d y\right)\d x = \int\limits_{a_2}^{b_2}\left(\int\limits_{a_1}^{b_1} f(x, y) \d x\right)\d y. \]
\end{theorem}

Całkę iterowaną często oznaczamy przenosząc symbol $\d x$ na początek, na przykład:
\[ \int\limits_{a_1}^{b_1} \left(\int\limits_{a_2}^{b_2} \left(\int\limits_{a_3}^{b_3} f(x, y, z) \d z\right)\d y\right)\d x = \int\limits_{a_1}^{b_1} \d x\int\limits_{a_2}^{b_2} \d y \int\limits_{a_3}^{b_3} f(x, y, z) \d z \]

\begin{definition}
    Funkcja charakterystyczna zbioru $D$ to funkcja
    \[ \chi_D(x, y) = \begin{cases}
        1, & (x, y) \in D \\
        0, & (x, y) \notin D
    \end{cases}. \]
\end{definition}

Korzystając z tej definicji, jeśli $D \in \RR^2$ jest zbiorem zawierającym się w pewnym prostokącie $P$, to
\[ \iint\limits_D f(x, y) \d x \d y = \iint\limits_P (f\chi_D)(x, y) \d x \d y. \]

\begin{definition}
    Obszar normalny (względem osi $OX$) to taki zbiór
    \[ D = \{(x, y) \in \RR^2 : x \in [a, b], y \in [\varphi(x), \psi(x)]\}, \]
    że funkcje $\varphi, \psi$ są ciągłe.
\end{definition}

Analogicznie możemy zdefiniować obszar normalny względem osi $OY$.

\begin{theorem}[zamiana całki powójnej na całkę iterowaną dla obszaru normalnego]
    Jeśli funkcja $f$ jest ciągła oraz
    \[ D = \{(x, y) \in \RR^2 : x \in [a, b], y \in [\varphi(x), \psi(x)]\} \]
    jest obszarem normalnym względem osi $OX$ tej funkcji, to
    \[ \iint\limits_D f(x, y) \d x \d y = \int\limits_a^b \d x \int\limits_{\varphi(x)}^{\psi(x)} f(x, y) \d y. \]
\end{theorem}

\begin{definition}
    Obszar regularny jest skończoną sumą obszarów normalnych o parami rozłącznych wnętrzach.
\end{definition}

\begin{example}
    Znaleźć pole obszaru $R$ ograniczonego krzywą $y = x^2$ i prostą $y = x + 2$.
\end{example}
\begin{solution}
    Najpierw liczymy punkty przecięcia krzywych: $(-1, 1), (2, 4)$. Możemy potraktować cały obszar $R$ jako normalny względem osi $OX$ otrzymując
    \[ [R] = \int\limits_{-1}^2 \d x \int\limits_{x^2}^{x+2} \d y = \int\limits_{-1}^2 (y + 2 - y^2) \d x = \left[\frac{y^2}{2} + 2y - \frac{y^3}{3}\right]_{-1}^2 = \frac{9}{2}. \]

    Alternatywnie, możemy podzielić $R$ na dwa obszary normalne, jak na rysunku.
    \begin{center}
        \begin{tikzpicture}[scale=0.8]
            \tkzInit[xmin=-1.5,ymin=-0.5,xmax=5,ymax=4.5]
            \tkzGrid
            \tkzDrawX
            \tkzDrawY
            \tkzDefPoint(-1,1){A}
            \tkzDefPoint(2,4){B}

            \draw[fill=MainColor1, opacity=.25] (-1,1) parabola bend (0,0) (2,4) -- cycle;
            \node[black] at (0.2,0.5) {\scriptsize $R_1$};
            \node[black] at (0.8,1.8) {\scriptsize $R_2$};

            \draw[BoxColor1,thick,domain=-1.1:2.1] plot (\x,{pow(\x, 2)});
            \draw[BoxColor2,thick,domain=-1.2:2.2] plot (\x,{\x+2});
            \draw[black,thick,domain=-1:1] plot (\x, 1);

            \tkzDrawPoints(A,B)
            \tkzLabelPoint[font=\scriptsize, below left](A){$(-1,1)$}
            \tkzLabelPoint[font=\scriptsize, right](B){$(2,4)$}
        \end{tikzpicture}
    \end{center}
    Obszar $R_1$ jest normalny względem osi $OY$:
    \[ [R_1] = \int\limits_0^1 \d y \int\limits_{-\sqrt{y}}^{\sqrt{y}} \d x = 2 \int\limits_0^1 \sqrt{y} \d y = \frac{4}{3}, \]
    podobnie jak obszar $R_2$:
    \[ [R_2] = \int\limits_1^4 \d y \int\limits_{y-2}^{\sqrt{y}} \d x = \int\limits_1^4 (\sqrt{y} - y + 2) \d y = \left[\frac{2y^{\frac{3}{2}}}{3} - \frac{y^2}{2} + 2y\right]_1^4 = \frac{14}{3} - \frac{1}{2}. \]
\end{solution}

\begin{theorem}[o zamianie zmiennych w całce wielokrotnej]
    \label{t:change variables in double integral}
    Dana jest funkcja $f : D \to \RR$, która jest ciągła na obszarze regularnym $D$ oraz przekształcenie $\varPhi : D' \to D$, gdzie
    \[ \varPhi : (x, y) \mapsto (x(u, v), (y(u, v))). \]
    Jeśli $\varPhi$ przekształca wnętrze obszaru regularnego $D'$ na wnętrze obszaru regularnego $D$ wznajemnie jednoznacznie (jest bijekcją), pochodne cząstkowe przekształcenia $\varPhi$ są ciągłe na pewnym zbiorze otwartym zawierającym obszar $D'$ oraz jakobian przekształcenia $J_\varPhi(u, v)$ jest niezerowy wewnątrz $D'$, to
    \[ \iint\limits_D f(x, y) \d x \d y = \iint\limits_{D'} f(x(u, v), y(u, v)) \cdot |J(u, v)| \d u \d v. \]
\end{theorem}

\begin{example}
    Obliczyć całkę
    \[ \iint\limits_D (x + y) \d x \d y \]
    po obszarze $D : 2 \leq 2x + y \leq 3, -1 \leq x - y \leq 1$.
\end{example}
\begin{solution}
    Możemy podstawić
    \[ \begin{cases} u = 2x + y \\ v = x - y \end{cases} \implies \begin{cases} x = \frac{u + v}{3} \\ y = \frac{u - 2v}{3} \end{cases}. \]
    Teraz obszar jest prostokątem, $D' = 2 \leq u \leq 3, -1 \leq v \leq 1$. Obliczmy jakobian
    \[ J(u, v) = \begin{vmatrix}
        \frac{\p x}{\p u} & \frac{\p x}{\p v} \\
        \frac{\p y}{\p u} & \frac{\p y}{\p v}
    \end{vmatrix} = \begin{vmatrix}
        \frac{1}{3} & \frac{1}{3} \\
        \frac{1}{3} & \frac{-2}{3}
    \end{vmatrix} = \frac{-2}{9} - \frac{1}{9} = \frac{-1}{3}. \]
    Wykorzystując twierdzenie o zamianie zmiennych (\ref{t:change variables in double integral}) mamy
    \begin{align*}
        \iint\limits_D (x + y) \d x \d y &= \iint\limits_{D'} \left(\frac{u + v}{3} + \frac{u - 2v}{3}\right) \cdot |J(u, v)| \d u \d v = \iint\limits_{D'} \frac{2u - v}{9} \d u \d v \\
        &= \frac{1}{9}\int\limits_{-1}^1 \d v \int\limits_2^3 (2u - v) \d u = \frac{1}{9}\int\limits_{-1}^1 (5 - v) \d v = \frac{10}{9}.
    \end{align*}
\end{solution}

\begin{example}
    Znaleźć objętość bryły ograniczonej przez paraboloidę $z = 4 - x^2 - y^2$, brzeg walca $(x - 1)^2 + y^2 = 1$ oraz płaszczyznę $z = 0$ (od dołu).
\end{example}
\begin{solution}
    Bryła ma objętość równą
    \[ \iint\limits_{D} 4 - x^2 - y^2, \]
    przy $D = \{(x, y) : (x - 1)^2 + y^2 \leq 1\}$. Możemy przejść do współrzędnych biegunowych\footnote{zobacz dodatek \ref{s:coordinate systems}}:
    \[ \iint\limits_{D} 4 - x^2 - y^2 \d x \d y = \iint\limits_{D'} (4 - r^2) r \d r \d \varphi, \]
    gdzie, skoro $x^2 + y^2 - 2x + 1 \leq 1 \implies r^2 \leq 2r\cos\varphi \implies r \leq 2\cos\varphi$,
    \[ D' = \{(r, \varphi) : \varphi \in [0, \pi], r \in [0, 2\cos\varphi] \}. \]
    Zbiór $D'$ jest obszarem normalnym, mamy więc
    \begin{align*}
        \iint\limits_{D'} (4r - r^3) \d r \d \varphi &= \int\limits_0^\pi \d \varphi \int\limits_0^{2\cos\varphi} 4r - r^3 \d r = \int\limits_0^\pi (8\cos^2\varphi - 4\cos^4\varphi) \d \varphi \\
        &= \left[\frac{5}{2}\varphi + \frac{5}{2}\sin\varphi\cos\varphi - \sin\varphi\cos^3\varphi\right]_0^\pi = \frac{5\pi}{2}.
    \end{align*}
\end{solution}