Podobnie do szeregów liczbowych, szeregi funkcyjne to para $((f_n(x))_{n\in\NN}, (S_n(x))_{n\in\NN})$: ciąg funkcyjny oraz ciąg sum częściowych ciągu funkcyjnego. Taki szereg jest zbieżny (punktowo / jednostajnie) do sumy szeregu $S$, jeśli ciąg $(S_n(x))$ jest zbieżny (częściowo / jednostajnie) do $S$.

Analogicznie do twierdzenia \ref{t:uniform convergence implies pointwise convergence}, warunkiem koniecznym zbieżności jednostajnej szeregu jest jego zbieżność punktowa.

Z kolei w analogii do twierdzenia \ref{t:necessary condition of convergence}, warunkiem koniecznym zbieżności (punktowej / jednostajnej) szeregu $\sum_{n=1}^\infty f_n(x)$ jest zbieżność (punktowa / jednostajna) ciągu funkcyjnego $(f_n(x))$ do zera, to znaczy
\[ \sum_{n=1}^\infty f_n(x) \rightarrow S \Longrightarrow f_n(x) \rightarrow 0 \equiv f \]
oraz
\[ \sum_{n=1}^\infty f_n(x) \rightrightarrows S \Longrightarrow f_n(x) \rightrightarrows 0 \equiv f. \]

\begin{theorem}[kryterium Weierstrassa]
    Jeśli istnieje taki ciąg $(a_n)$, że dla każdego $n \in \NN$ i dla każdego $x \in X \subset \RR$ mamy nierówność
    \[ |f_n(x)| \leq a_n \]
    oraz szereg $\sum_{n=1}^\infty a_n$ jest zbieżny, to szereg funkcyjny
    \[ \sum_{n=1}^\infty f_n(x) \]
    jest jednostajnie zbieżny na $X$.
\end{theorem}

Zachodzi twierdzenie o ciągłości, analogiczne do twierdzenia \ref{t:continuous limit}.

\begin{theorem}
    \label{t:continuous series}
    Jeśli szereg $\sum_{n=1}^\infty f_n(x)$ jest szeregiem funkcji ciągłych i jest jednostajnie zbieżny $\sum_{n=1}^\infty f_n(x) \rightrightarrows S(x)$, to funkcja $S$ jest ciągła.
\end{theorem}

\begin{example}
    Zbadaj zbieżność punktową i jednostajną szeregu
    \[ \sum_{n=1}^\infty x^n(1-x) \]
    na przedziale $[0,1]$.
\end{example}
\begin{solution}
    Dla $x \in [0, 1)$ mamy:
    \[ \sum_{n=1}^\infty x^n(1-x) = x(1-x)\frac{1}{1-x} = x, \]
    natomiast dla $x = 1$ mamy
    \[ \sum_{n=1}^\infty x^n(1-x) = \sum_{n=1}^\infty 1^n \cdot 0 = 0, \]
    więc szereg jest zbieżny punktowo. Funkcja
    \[ S(x) = \begin{cases}x, & \text{dla } x \in [0, 1) \\ 0, & \text{dla } x = 1 \end{cases}, \]
    do której dany szereg zbiega nie jest ciągła, a funkcje $f_n(x) = x^n(1-x)$ są ciągłe, więc, na mocy twierdzenia \ref{t:continuous series}, szereg nie zbiega jednostajnie.
\end{solution}

\begin{example}
    Zbadaj zbieżność punktową i jednostajną szeregu
    \[ \sum_{n=1}^\infty \frac{nx}{1+n^4x^2} \]
    na przedziale $[1,\infty)$.
\end{example}
\begin{solution}
    Dla każdego $x \in [1, \infty]$ oraz $n \in \NN$ mamy
    \[ \left|\frac{nx}{1+n^4x^2}\right| = \frac{nx}{1+n^4x^2} \leq \frac{nx}{n^4x^2} = \frac{1}{n^3x} \leq \frac{1}{n^3}, \]
    więc, na mocy kryterium Weierstrassa, dany szereg jest jednostajnie zbieżny, bo szereg harmoniczny rzędu $3$ jest zbieżny.
\end{solution}

\begin{example}
    Zbadaj obszar zbieżności\footnote{czyli zbiór punktów, w których szereg jest zbieżny} punktowej oraz zbieżność jednostajną szeregu
    \[ \sum_{n=1}^\infty \frac{x^2}{e^{nx}}. \]
\end{example}
\begin{solution}
    Możemy od razu stwierdzić, że dla $x = 0$ otrzymamy szereg ciąg zer, który oczywiście jest (jednostajnie) zbieżny do zera. Możemy potraktować $x$ jako parametr, wtedy zamiast szeregu funkcyjnego będziemy mieć szereg liczbowy, którego zbieżność możemy pokazać z kryterium d'Alemberta:
    \[ g = \lim_{n\to\infty} \frac{x^2}{e^{x(n+1)}}\frac{e^{xn}}{x^2} = \lim_{n\to\infty}\frac{1}{e^x} = \frac{1}{e^x}. \]
    Szereg jest więc zbieżny dla każdego $x > 0$ i rozbieżny dla każdego $x < 0$. Ostatecznie, obszar zbieżności punktowej danego szeregu funkcyjnego to $[0,\infty)$.

    Zajmijmy się teraz zbieżnością jednostajną. Oczywiście można by ją wykazywać przez znalezienie ciągu sum częściowych, a następnie skorzystanie z twierdzenia \ref{t:uniform convergence iff metric = 0}, ale możemy też skorzystać z kryterium Weierstrassa, chociaż w dosyć nieoczywisty sposób.

    Znajdźmy najpierw supremum ciągu $a_n = \frac{x^2}{e^{nx}}$. Możemy znaleźć miejsca zerowe pochodnej:
    \[ \ddx \frac{x^2}{e^{nx}} = \frac{2x(e^{nx}) - x^2(ne^{nx})}{e^{2nx}} = \frac{x(2 - xn)}{e^{nx}} = 0 \iff x \in \left\{0, \frac{2}{n}\right\}. \]
    Szkicując wykres pochodnej przekonamy się, że funkcja $a_n(x)$ osiąga maksimum w $x = \frac{2}{n}$, więc
    \[ a_n(x) \leq a_n\left(\tfrac{2}{n}\right) = \frac{\left(\frac{2}{n}\right)^2}{e^2} = \frac{4}{e^2n^2}. \]
    Szereg $\sum_{n=1}^\infty \frac{4}{e^2n^2}$ jest zbieżny (ponieważ jest harmoniczny rzędu $2$), więc możemy użyć kryterium Weierstrassa udowadniając, że dany szereg funkcyjny jest jednostajnie zbieżny.
\end{solution}

Zachodzą również twierdzenia o różniczkowalności i całkowalności, analogiczne do twierdzeń \ref{t:differentiable limit} i \ref{t:integrable limit}.

\begin{theorem}
    \label{t:differentiable series}
    Niech $(f_n(x))$ będzie ciągiem funkcji różniczkowalnych. Jeśli szereg $\sum_{n=1}^\infty f_n(x)$ jest zbieżny na $X$, a szereg $\sum_{n=1}^\infty f_n'(x)$ jest jednostajnie zbieżny na $X$, to dla każdego $x \in X$ zachodzi
    \[ \left(\sum_{n=1}^\infty f_n(x)\right)' = \sum_{n=1}^\infty f_n'(x). \]
\end{theorem}

\begin{theorem}
    \label{t:integrable series}
    Niech $(f_n(x))$ będzie ciągiem funkcji całkowalnych. Jeśli szereg $\sum_{n=1}^\infty f_n(x)$ jest jednostajnie zbieżny na $X$, to dla każdych $x_1, x_2 \in X$ zachodzi
    \[ \int_{x_1}^{x_2}\left(\sum_{n=1}^\infty f_n(x)\right)\d x = \sum_{n=1}^\infty \left(\int_{x_1}^{x_2}f_n(x)\d x\right). \]
\end{theorem}

\subsection{Szeregi potęgowe}
\begin{definition}
    \label{d:power series}
    Szereg potęgowy o środku w punkcie $c$ to szereg funkcyjny
    \[ \sum_{n=1}^\infty a_n(x - c)^n, \]
    gdzie $a_n, x, c \in \CC$.
\end{definition}

\begin{theorem}
    Jeśli szereg potęgowy
    \[ \sum_{n=1}^\infty a_n(x - c)^n \]
    jest zbieżny dla pewnego $x_1$, to jest zbieżny dla wszystkich $x_2$ takich, że
    \[ |x_2 - c| < |x_1 - c|, \]
    a jeśli nie jest zbieżny dla pewnego $x_1$, to nie jest zbieżny dla wszystkich $x_2$ takich, że
    \[ |x_2 - c| > |x_1 - c|. \]
\end{theorem}

Powyższe twierdzenie każe nam podzielić płaszczyznę zespoloną (względem danego szeregu potęgowego) na trzy rozłączne zbiory. Formalnie, jeśli weźmiemy
\[ r = \sup\left\{|x - c| : \text{ szereg } \sum_{n=1}^\infty a_n(x - c)^n \text{ jest zbieżny}\right\}, \]
to zbiór
\[ \{x \in \CC : |x - x_0| < r\} \]
nazwiemy \vocab{kołem zbieżności}. Dla wszystkich elementów z tego zbioru dany szereg jest zbieżny. Dla elementów na brzegu tego koła zbieżność jest nieokreślona, a dla elementów poza nim dany szereg nie jest zbieżny. Liczba $r$ to \vocab{promień zbieżności}. Dla $x=c$ dany szereg jest zbieżny.

\begin{remark*}
    Jeśli przyjmiemy w definicji szeregu potęgowego (\ref{d:power series}), że $a_n, x, c \in \RR$, to koło zbieżności staje się \vocab{przedziałem zbieżności}, a nieokreśloną zbieżność mamy tylko dla dwóch elementów: $c - r$ oraz $c + r$.
\end{remark*}

\vocab{Obszarem zbieżności} nazywamy zbiór będący sumą koła zbieżności oraz zbioru elementów z jego brzegu, dla których dany szereg potęgowy jest zbieżny.

\begin{theorem}[Cauchy'ego-Hadamarda]
    \label{t:Cauchy-Hadamard}
    Promień zbieżności jest dany jako
    \[ r = \frac{1}{\limsup\limits_{n\to\infty}\sqrt[n]{|a_n|}}, \]
    gdzie $r = \frac{1}{0}$ interpretujemy jako $r = \infty$, a $r = \frac{1}{\infty}$ jako $r = 0$.
\end{theorem}

Można podać dwa słabsze twierdzenia, które jednak często łatwiej jest stosować:
\[ r = \frac{1}{\lim\limits_{n\to\infty} \left|\frac{a_{n+1}}{a_n}\right|} \hspace{1em}\Longrightarrow\hspace{1em} r = \frac{1}{\lim\limits_{n\to\infty}\sqrt[n]{|a_n|}} \hspace{1em}\Longrightarrow\hspace{1em} (\ref{t:Cauchy-Hadamard}). \]

Mówimy, że ciąg (szereg) funkcyjny jest \vocab{niemal jednostajnie zbieżny} na przedziale $(a, b)$ jeśli jest jednostajnie zbieżny na każdym przedziale $[c, d] \in (a, b)$.

\begin{fact}
    Jeśli szereg potęgowy jest zbieżny w $(c-r, c+r)$, to jest bezwzględnie zbieżny w $(c-r, c+r)$ oraz niemal jednostajnie zbieżny w $(c-r, c+r)$.
\end{fact}

\begin{fact}
    Jeśli szereg potęgowy jest zbieżny w $(c-r, c+r)$ do $S(x)$, to funkcja $S(x)$ jest ciągła, różniczkowalna i całkowalna w $(c-r, c+r)$. Prawdziwe dla szeregów potęgowych są również tezy twierdzeń \ref{t:differentiable series} i \ref{t:integrable series}.
\end{fact}

\begin{theorem}[Abela]
    \label{t:Abel}
    Niech $\sum_{n=1}^\infty a_n(x - c)^n$ będzie szeregiem potęgowym zbieżnym do $S(x)$ o promieniu zbieżności równym $r$. Jeśli ten szereg jest zbieżny dla $x_1 = c - r$ oraz istnieje granica $\lim\limits_{x\to x_1^+}S(x)$, to
    \[ \lim_{x\to x_1^+}S(x) = S(x_1), \]
    czyli funkcja $S(x)$ jest prawostronnie ciągła w $x = c - r$. Analogicznie, jeśli szereg jest zbieżny dla $x_2 = c + r$ oraz istnieje granica $\lim\limits_{x\to x_2^-}S(x)$, to
    \[ \lim_{x\to x_2^-}S(x) = S(x_2), \]
    czyli funkcja $S(x)$ jest lewostronnie ciągła w $x = c + r$.
\end{theorem}

\begin{example}
    Znajdź sumę szeregu
    \[ \sum_{n=1}^\infty\frac{(n+1)(x+2)^n}{2^n} \]
    w każdym punkcie obszaru zbieżności.
\end{example}
\begin{solution}
    Stosując twierdzenie Cauchy'ego-Hadamarda (\ref{t:Cauchy-Hadamard}) możemy obliczyć promień zbieżności danego szeregu
    \[ r = \frac{1}{\lim\limits_{n\to\infty}\sqrt[n]{\frac{n+1}{2^n}}} = \frac{1}{\frac{1}{2}} = 2, \]
    tak więc przedział zbieżności to $(-4, 0)$.
    Dla $x = -4$ mamy
    \[ \sum_{n=1}^\infty\frac{(n+1)(-2)^n}{2^n} = \sum_{n=1}^\infty(-1)^n(n+1) \text{ -- rozbieżny, nie spełnia warunku koniecznego}, \]
    a dla $x = 0$
    \[ \sum_{n=1}^\infty\frac{(n+1)2^n}{2^n} = \sum_{n=1}(n+1) \text{ -- rozbieżny, nie spełnia warunku koniecznego}. \]
    Obszarem zbieżności jest więc przedział $(-4, 0)$. Policzmy teraz sumę. Dla każdego $x \in (-4, 0)$ mamy
    \begin{align*}
        S(x) &= \sum_{n=1}^\infty\frac{(n+1)(x+2)^n}{2^n} = \sum_{n=1}^\infty \left(\frac{(x+2)^{n+1}}{2^n}\right)' \overset{(\ref{t:differentiable series})}{=} \left(\sum_{n=1}^\infty\frac{(x+2)^{n+1}}{2^n}\right)' \\
        &= \left(\frac{(x+2)^2}{2}\frac{1}{1 - \frac{x+2}{2}}\right)' = \left(\frac{(x+2)^2}{-x}\right)' = \frac{2x(x+2) + (x+2)^2}{x^2} = \frac{4 - x^2}{x^2}.
    \end{align*}
\end{solution}

\begin{example}
    Znajdź sumę szeregu
    \[ \sum_{n=0}^\infty \frac{2^n(x - \frac{1}{2})^n}{n+1} \]
    w każdym punkcie obszaru zbieżności.
\end{example}
\begin{solution}
    Stosując twierdzenie Cauchy'ego-Hadamarda (\ref{t:Cauchy-Hadamard}) możemy obliczyć promień zbieżności danego szeregu
    \[ r = \frac{1}{\lim\limits_{n\to\infty}\sqrt[n]{\frac{2^n}{n+1}}} = \frac{1}{2}, \]
    tak więc przedział zbieżności to $(0, 1)$.
    Dla $x = 0$ mamy
    \[ \sum_{n=0}^\infty \frac{2^n\left(-\frac{1}{2}\right)^n}{n+1} = \sum_{n=0}^\infty \frac{(-1)^n}{n+1} \text{ -- zbieżny z kryterium Leibniza}, \]
    a dla $x = 1$
    \[ \sum_{n=0}^\infty \frac{2^n\left(\frac{1}{2}\right)^n}{n+1} = \sum_{n=0}^\infty \frac{1}{n+1} \text{ -- rozbieżny z kryterium ilorazowego}. \]
    Obszarem zbieżności jest więc przedział $[0, 1)$. Policzmy teraz sumę. Dla $x = \frac{1}{2}$ mamy
    \[ S(\tfrac{1}{2}) = \sum_{n=0}^\infty \frac{2^n0^n}{n+1} = 1 + 0 + 0 + \cdots = 1. \]
    Dla pozostałych $x$ zapiszemy
    \[ S(x) = \sum_{n=0}^\infty \frac{2^n(x - \frac{1}{2})^n}{n+1} = \frac{1}{x - \frac{1}{2}}\sum_{n=0}^\infty \frac{2^n(x - \frac{1}{2})^{n+1}}{n+1} = \frac{1}{x - \frac{1}{2}}\sum_{n=0}^\infty \int_{\frac{1}{2}}^x 2^n\left(t-\frac{1}{2}\right)^n\d t. \]
    Szeregi potęgowe są niemal jednostajnie zbieżne w swoim przedziale zbieżności, więc dla $x \in (0, 1)$ możemy zamienić znaki sumy i całki (twierdzenie \ref{t:integrable series})
    \begin{align*}
        S(x) &= \frac{1}{x - \frac{1}{2}}\int_{\frac{1}{2}}^x \sum_{n=0}^\infty 2^n\left(t-\frac{1}{2}\right)^n\d t = \frac{1}{x - \frac{1}{2}}\int_{\frac{1}{2}}^x \frac{1}{1 - 2(t - \frac{1}{2})}\d t \\
        &= \frac{1}{x - \frac{1}{2}}\int_{\frac{1}{2}}^x \frac{1}{2 - 2t}\d t = \frac{1}{x - \frac{1}{2}}\left[-\frac{1}{2}\ln(1 - t)\right]_{\frac{1}{2}}^x = \frac{1}{1 - 2x}\left(\ln(1 - x) - \ln{\frac{1}{2}}\right) \\
        &= \frac{\ln(2 - 2x)}{1 - 2x}.
    \end{align*}
    Z twierdzenia Abela (\ref{t:Abel}) wynika, że
    \[ S(0) = \lim_{x\to 0^+}\frac{\ln(2 - 2x)}{1 - 2x} = \ln(2), \]
    więc ostatecznie mamy
    \[ S(x) = \begin{cases}1, & \text{dla } x = \tfrac{1}{2} \\ \frac{\ln(2 - 2x)}{1 - 2x}, & \text{dla } x \in [0, 1)\setminus \{\tfrac{1}{2}\} \end{cases}. \]
\end{solution}

\begin{example}
    Znajdź sumę szeregu liczbowego
    \[ 1 - \frac{1}{3} + \frac{1}{5} - \frac{1}{7} + \ldots. \]
\end{example}
\begin{solution}
    Weźmy szereg funkcyjny
    \[ S(x) = \sum_{n=0}^\infty \frac{(-1)^n}{2n+1}x^{2n+1}. \]
    Wartość $S(1)$ jest szukaną sumą, jeśli tylko szereg jest zbieżny w tym punkcie. Niech $t = x^2$. Stosując twierdzenie Cauchy'ego-Hadamarda (\ref{t:Cauchy-Hadamard}) możemy obliczyć promień zbieżności szeregu:
    \[ r_t = \frac{1}{\lim\limits_{n\to\infty}\frac{2n+1}{2n+3}} = 1, \]
    tak więc szereg zbiega, gdy $t \in (-1, 1) \implies x \in (-1, 1)$. W punktach $x = -1$ i $x = 1$ szereg również jest zbieżny, co można pokazać z kryterium Leibniza.

    Policzmy teraz sumę (dla przedziału zbieżności $(-1, 1)$):
    \begin{align*}
        S(x) &= \sum_{n=0}^\infty \frac{(-1)^n}{2n+1}x^{2n+1} = \sum_{n=0}^\infty\int_0^x (-1)^n u^{2n} \d u = \int_0^x\sum_{n=0}^\infty (-1)^n u^{2n} \d u \\
        &= \int_0^x\sum_{n=0}^\infty (-u^2)^n \d u = \int_0^x \frac{1}{1 + u^2} \d u = \Big[\arctan(u)\Big]_0^x = \arctan(x).
    \end{align*}

    Skoro w $x = 1$ ten szereg też jest zbieżny, to z twierdzenia Abela (\ref{t:Abel}) mamy
    \[ S(1) = \lim_{x\to 1}\arctan(x) = \arctan(1) = \frac{\pi}{4}. \]
\end{solution}